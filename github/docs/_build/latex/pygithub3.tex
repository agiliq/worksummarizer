% Generated by Sphinx.
\def\sphinxdocclass{report}
\documentclass[letterpaper,10pt,english]{sphinxmanual}
\usepackage[utf8]{inputenc}
\DeclareUnicodeCharacter{00A0}{\nobreakspace}
\usepackage[T1]{fontenc}
\usepackage{babel}
\usepackage{times}
\usepackage[Bjarne]{fncychap}
\usepackage{longtable}
\usepackage{sphinx}
\usepackage{multirow}


\title{pygithub3 Documentation}
\date{May 28, 2012}
\release{0.3}
\author{David Medina}
\newcommand{\sphinxlogo}{}
\renewcommand{\releasename}{Release}
\makeindex

\makeatletter
\def\PYG@reset{\let\PYG@it=\relax \let\PYG@bf=\relax%
    \let\PYG@ul=\relax \let\PYG@tc=\relax%
    \let\PYG@bc=\relax \let\PYG@ff=\relax}
\def\PYG@tok#1{\csname PYG@tok@#1\endcsname}
\def\PYG@toks#1+{\ifx\relax#1\empty\else%
    \PYG@tok{#1}\expandafter\PYG@toks\fi}
\def\PYG@do#1{\PYG@bc{\PYG@tc{\PYG@ul{%
    \PYG@it{\PYG@bf{\PYG@ff{#1}}}}}}}
\def\PYG#1#2{\PYG@reset\PYG@toks#1+\relax+\PYG@do{#2}}

\expandafter\def\csname PYG@tok@gd\endcsname{\def\PYG@tc##1{\textcolor[rgb]{0.63,0.00,0.00}{##1}}}
\expandafter\def\csname PYG@tok@gu\endcsname{\let\PYG@bf=\textbf\def\PYG@tc##1{\textcolor[rgb]{0.50,0.00,0.50}{##1}}}
\expandafter\def\csname PYG@tok@gt\endcsname{\def\PYG@tc##1{\textcolor[rgb]{0.00,0.25,0.82}{##1}}}
\expandafter\def\csname PYG@tok@gs\endcsname{\let\PYG@bf=\textbf}
\expandafter\def\csname PYG@tok@gr\endcsname{\def\PYG@tc##1{\textcolor[rgb]{1.00,0.00,0.00}{##1}}}
\expandafter\def\csname PYG@tok@cm\endcsname{\let\PYG@it=\textit\def\PYG@tc##1{\textcolor[rgb]{0.25,0.50,0.56}{##1}}}
\expandafter\def\csname PYG@tok@vg\endcsname{\def\PYG@tc##1{\textcolor[rgb]{0.73,0.38,0.84}{##1}}}
\expandafter\def\csname PYG@tok@m\endcsname{\def\PYG@tc##1{\textcolor[rgb]{0.13,0.50,0.31}{##1}}}
\expandafter\def\csname PYG@tok@mh\endcsname{\def\PYG@tc##1{\textcolor[rgb]{0.13,0.50,0.31}{##1}}}
\expandafter\def\csname PYG@tok@cs\endcsname{\def\PYG@tc##1{\textcolor[rgb]{0.25,0.50,0.56}{##1}}\def\PYG@bc##1{\setlength{\fboxsep}{0pt}\colorbox[rgb]{1.00,0.94,0.94}{\strut ##1}}}
\expandafter\def\csname PYG@tok@ge\endcsname{\let\PYG@it=\textit}
\expandafter\def\csname PYG@tok@vc\endcsname{\def\PYG@tc##1{\textcolor[rgb]{0.73,0.38,0.84}{##1}}}
\expandafter\def\csname PYG@tok@il\endcsname{\def\PYG@tc##1{\textcolor[rgb]{0.13,0.50,0.31}{##1}}}
\expandafter\def\csname PYG@tok@go\endcsname{\def\PYG@tc##1{\textcolor[rgb]{0.19,0.19,0.19}{##1}}}
\expandafter\def\csname PYG@tok@cp\endcsname{\def\PYG@tc##1{\textcolor[rgb]{0.00,0.44,0.13}{##1}}}
\expandafter\def\csname PYG@tok@gi\endcsname{\def\PYG@tc##1{\textcolor[rgb]{0.00,0.63,0.00}{##1}}}
\expandafter\def\csname PYG@tok@gh\endcsname{\let\PYG@bf=\textbf\def\PYG@tc##1{\textcolor[rgb]{0.00,0.00,0.50}{##1}}}
\expandafter\def\csname PYG@tok@ni\endcsname{\let\PYG@bf=\textbf\def\PYG@tc##1{\textcolor[rgb]{0.84,0.33,0.22}{##1}}}
\expandafter\def\csname PYG@tok@nl\endcsname{\let\PYG@bf=\textbf\def\PYG@tc##1{\textcolor[rgb]{0.00,0.13,0.44}{##1}}}
\expandafter\def\csname PYG@tok@nn\endcsname{\let\PYG@bf=\textbf\def\PYG@tc##1{\textcolor[rgb]{0.05,0.52,0.71}{##1}}}
\expandafter\def\csname PYG@tok@no\endcsname{\def\PYG@tc##1{\textcolor[rgb]{0.38,0.68,0.84}{##1}}}
\expandafter\def\csname PYG@tok@na\endcsname{\def\PYG@tc##1{\textcolor[rgb]{0.25,0.44,0.63}{##1}}}
\expandafter\def\csname PYG@tok@nb\endcsname{\def\PYG@tc##1{\textcolor[rgb]{0.00,0.44,0.13}{##1}}}
\expandafter\def\csname PYG@tok@nc\endcsname{\let\PYG@bf=\textbf\def\PYG@tc##1{\textcolor[rgb]{0.05,0.52,0.71}{##1}}}
\expandafter\def\csname PYG@tok@nd\endcsname{\let\PYG@bf=\textbf\def\PYG@tc##1{\textcolor[rgb]{0.33,0.33,0.33}{##1}}}
\expandafter\def\csname PYG@tok@ne\endcsname{\def\PYG@tc##1{\textcolor[rgb]{0.00,0.44,0.13}{##1}}}
\expandafter\def\csname PYG@tok@nf\endcsname{\def\PYG@tc##1{\textcolor[rgb]{0.02,0.16,0.49}{##1}}}
\expandafter\def\csname PYG@tok@si\endcsname{\let\PYG@it=\textit\def\PYG@tc##1{\textcolor[rgb]{0.44,0.63,0.82}{##1}}}
\expandafter\def\csname PYG@tok@s2\endcsname{\def\PYG@tc##1{\textcolor[rgb]{0.25,0.44,0.63}{##1}}}
\expandafter\def\csname PYG@tok@vi\endcsname{\def\PYG@tc##1{\textcolor[rgb]{0.73,0.38,0.84}{##1}}}
\expandafter\def\csname PYG@tok@nt\endcsname{\let\PYG@bf=\textbf\def\PYG@tc##1{\textcolor[rgb]{0.02,0.16,0.45}{##1}}}
\expandafter\def\csname PYG@tok@nv\endcsname{\def\PYG@tc##1{\textcolor[rgb]{0.73,0.38,0.84}{##1}}}
\expandafter\def\csname PYG@tok@s1\endcsname{\def\PYG@tc##1{\textcolor[rgb]{0.25,0.44,0.63}{##1}}}
\expandafter\def\csname PYG@tok@gp\endcsname{\let\PYG@bf=\textbf\def\PYG@tc##1{\textcolor[rgb]{0.78,0.36,0.04}{##1}}}
\expandafter\def\csname PYG@tok@sh\endcsname{\def\PYG@tc##1{\textcolor[rgb]{0.25,0.44,0.63}{##1}}}
\expandafter\def\csname PYG@tok@ow\endcsname{\let\PYG@bf=\textbf\def\PYG@tc##1{\textcolor[rgb]{0.00,0.44,0.13}{##1}}}
\expandafter\def\csname PYG@tok@sx\endcsname{\def\PYG@tc##1{\textcolor[rgb]{0.78,0.36,0.04}{##1}}}
\expandafter\def\csname PYG@tok@bp\endcsname{\def\PYG@tc##1{\textcolor[rgb]{0.00,0.44,0.13}{##1}}}
\expandafter\def\csname PYG@tok@c1\endcsname{\let\PYG@it=\textit\def\PYG@tc##1{\textcolor[rgb]{0.25,0.50,0.56}{##1}}}
\expandafter\def\csname PYG@tok@kc\endcsname{\let\PYG@bf=\textbf\def\PYG@tc##1{\textcolor[rgb]{0.00,0.44,0.13}{##1}}}
\expandafter\def\csname PYG@tok@c\endcsname{\let\PYG@it=\textit\def\PYG@tc##1{\textcolor[rgb]{0.25,0.50,0.56}{##1}}}
\expandafter\def\csname PYG@tok@mf\endcsname{\def\PYG@tc##1{\textcolor[rgb]{0.13,0.50,0.31}{##1}}}
\expandafter\def\csname PYG@tok@err\endcsname{\def\PYG@bc##1{\setlength{\fboxsep}{0pt}\fcolorbox[rgb]{1.00,0.00,0.00}{1,1,1}{\strut ##1}}}
\expandafter\def\csname PYG@tok@kd\endcsname{\let\PYG@bf=\textbf\def\PYG@tc##1{\textcolor[rgb]{0.00,0.44,0.13}{##1}}}
\expandafter\def\csname PYG@tok@ss\endcsname{\def\PYG@tc##1{\textcolor[rgb]{0.32,0.47,0.09}{##1}}}
\expandafter\def\csname PYG@tok@sr\endcsname{\def\PYG@tc##1{\textcolor[rgb]{0.14,0.33,0.53}{##1}}}
\expandafter\def\csname PYG@tok@mo\endcsname{\def\PYG@tc##1{\textcolor[rgb]{0.13,0.50,0.31}{##1}}}
\expandafter\def\csname PYG@tok@mi\endcsname{\def\PYG@tc##1{\textcolor[rgb]{0.13,0.50,0.31}{##1}}}
\expandafter\def\csname PYG@tok@kn\endcsname{\let\PYG@bf=\textbf\def\PYG@tc##1{\textcolor[rgb]{0.00,0.44,0.13}{##1}}}
\expandafter\def\csname PYG@tok@o\endcsname{\def\PYG@tc##1{\textcolor[rgb]{0.40,0.40,0.40}{##1}}}
\expandafter\def\csname PYG@tok@kr\endcsname{\let\PYG@bf=\textbf\def\PYG@tc##1{\textcolor[rgb]{0.00,0.44,0.13}{##1}}}
\expandafter\def\csname PYG@tok@s\endcsname{\def\PYG@tc##1{\textcolor[rgb]{0.25,0.44,0.63}{##1}}}
\expandafter\def\csname PYG@tok@kp\endcsname{\def\PYG@tc##1{\textcolor[rgb]{0.00,0.44,0.13}{##1}}}
\expandafter\def\csname PYG@tok@w\endcsname{\def\PYG@tc##1{\textcolor[rgb]{0.73,0.73,0.73}{##1}}}
\expandafter\def\csname PYG@tok@kt\endcsname{\def\PYG@tc##1{\textcolor[rgb]{0.56,0.13,0.00}{##1}}}
\expandafter\def\csname PYG@tok@sc\endcsname{\def\PYG@tc##1{\textcolor[rgb]{0.25,0.44,0.63}{##1}}}
\expandafter\def\csname PYG@tok@sb\endcsname{\def\PYG@tc##1{\textcolor[rgb]{0.25,0.44,0.63}{##1}}}
\expandafter\def\csname PYG@tok@k\endcsname{\let\PYG@bf=\textbf\def\PYG@tc##1{\textcolor[rgb]{0.00,0.44,0.13}{##1}}}
\expandafter\def\csname PYG@tok@se\endcsname{\let\PYG@bf=\textbf\def\PYG@tc##1{\textcolor[rgb]{0.25,0.44,0.63}{##1}}}
\expandafter\def\csname PYG@tok@sd\endcsname{\let\PYG@it=\textit\def\PYG@tc##1{\textcolor[rgb]{0.25,0.44,0.63}{##1}}}

\def\PYGZbs{\char`\\}
\def\PYGZus{\char`\_}
\def\PYGZob{\char`\{}
\def\PYGZcb{\char`\}}
\def\PYGZca{\char`\^}
\def\PYGZam{\char`\&}
\def\PYGZlt{\char`\<}
\def\PYGZgt{\char`\>}
\def\PYGZsh{\char`\#}
\def\PYGZpc{\char`\%}
\def\PYGZdl{\char`\$}
\def\PYGZti{\char`\~}
% for compatibility with earlier versions
\def\PYGZat{@}
\def\PYGZlb{[}
\def\PYGZrb{]}
\makeatother

\begin{document}

\maketitle
\tableofcontents
\phantomsection\label{index::doc}


\textbf{pygithub3} is a Github APIv3 python wrapper.

You can consume the API with several {\hyperref[services::doc]{\emph{Services}}} (users, repos...) like
you see in \href{http://developer.github.com}{Github API v3 documentation}.

When you do an API request, \textbf{pygithub3} map the result into {\hyperref[resources::doc]{\emph{Resources}}}
which can do its own related requests in future releases.


\chapter{Fast sample}
\label{index:documentation-overview}\label{index:fast-sample}
\begin{Verbatim}[commandchars=\\\{\}]
\PYG{k+kn}{from} \PYG{n+nn}{pygithub3} \PYG{k+kn}{import} \PYG{n}{Github}

\PYG{n}{gh} \PYG{o}{=} \PYG{n}{Github}\PYG{p}{(}\PYG{n}{login}\PYG{o}{=}\PYG{l+s}{'}\PYG{l+s}{octocat}\PYG{l+s}{'}\PYG{p}{,} \PYG{n}{password}\PYG{o}{=}\PYG{l+s}{'}\PYG{l+s}{password}\PYG{l+s}{'}\PYG{p}{)}

\PYG{n}{octocat} \PYG{o}{=} \PYG{n}{gh}\PYG{o}{.}\PYG{n}{users}\PYG{o}{.}\PYG{n}{get}\PYG{p}{(}\PYG{p}{)} \PYG{c}{\PYGZsh{} Auth required}
\PYG{n}{copitux} \PYG{o}{=} \PYG{n}{gh}\PYG{o}{.}\PYG{n}{users}\PYG{o}{.}\PYG{n}{get}\PYG{p}{(}\PYG{l+s}{'}\PYG{l+s}{copitux}\PYG{l+s}{'}\PYG{p}{)}
\PYG{c}{\PYGZsh{} copitux = \PYGZlt{}User (copitux)\PYGZgt{}}

\PYG{n}{octocat\PYGZus{}followers} \PYG{o}{=} \PYG{n}{gh}\PYG{o}{.}\PYG{n}{users}\PYG{o}{.}\PYG{n}{followers}\PYG{o}{.}\PYG{n}{list}\PYG{p}{(}\PYG{p}{)}\PYG{o}{.}\PYG{n}{all}\PYG{p}{(}\PYG{p}{)}
\PYG{n}{copitux\PYGZus{}followers} \PYG{o}{=} \PYG{n}{gh}\PYG{o}{.}\PYG{n}{users}\PYG{o}{.}\PYG{n}{followers}\PYG{o}{.}\PYG{n}{list}\PYG{p}{(}\PYG{l+s}{'}\PYG{l+s}{copitux}\PYG{l+s}{'}\PYG{p}{)}\PYG{o}{.}\PYG{n}{all}\PYG{p}{(}\PYG{p}{)}
\PYG{c}{\PYGZsh{} copitux\PYGZus{}followers = [\PYGZlt{}User (ahmontero)\PYGZgt{}, \PYGZlt{}User (alazaro)\PYGZgt{}, ...]}
\end{Verbatim}


\chapter{Others}
\label{index:others}
You must apologize my English level. I'm trying to do my best


\section{Installation}
\label{installation:installation}\label{installation::doc}
\begin{Verbatim}[commandchars=\\\{\}]
pip install pygithub3

or

easy\_install pygithub3
\end{Verbatim}


\subsection{Dependencies}
\label{installation:dependencies}

\subsubsection{Required}
\label{installation:required}
This library depends only on the \href{http://docs.python-requests.org/en/v0.10.6/index.html}{requests} module.

If you install \code{pygithub3} with \code{pip} all is done. This is the best option.


\subsubsection{Optional}
\label{installation:optional}
The test suite uses \href{http://readthedocs.org/docs/nose/en/latest}{nose}, \href{http://pypi.python.org/pypi/mock}{mock}, and \href{http://pypi.python.org/pypi/unittest2}{unittest2} (python 2.6). Compiling
the documentation requires \href{http://sphinx.pocoo.org/}{sphinx}.

Install all of these by running \code{pip install -r test\_requirements.txt}.  Then
just run \code{nosetests} to run the tests.


\section{Github}
\label{github::doc}\label{github:github}\label{github:sphinx}
This is the main entrance of \textbf{pygithub3}
\index{Github (class in pygithub3)}

\begin{fulllineitems}
\phantomsection\label{github:pygithub3.Github}\pysiglinewithargsret{\strong{class }\code{pygithub3.}\bfcode{Github}}{\emph{**config}}{}
You can preconfigure all services globally with a \code{config} dict. See
{\hyperref[services:pygithub3.services.base.Service]{\code{Service}}}

Example:

\begin{Verbatim}[commandchars=\\\{\}]
\PYG{n}{gh} \PYG{o}{=} \PYG{n}{Github}\PYG{p}{(}\PYG{n}{user}\PYG{o}{=}\PYG{l+s}{'}\PYG{l+s}{kennethreitz}\PYG{l+s}{'}\PYG{p}{,} \PYG{n}{token}\PYG{o}{=}\PYG{l+s}{'}\PYG{l+s}{ABC...}\PYG{l+s}{'}\PYG{p}{,} \PYG{n}{repo}\PYG{o}{=}\PYG{l+s}{'}\PYG{l+s}{requests}\PYG{l+s}{'}\PYG{p}{)}
\end{Verbatim}
\index{gists (pygithub3.Github attribute)}

\begin{fulllineitems}
\phantomsection\label{github:pygithub3.Github.gists}\pysigline{\bfcode{gists}}
{\hyperref[gists:gists-service]{\emph{Gists service}}}

\end{fulllineitems}

\index{remaining\_requests (pygithub3.Github attribute)}

\begin{fulllineitems}
\phantomsection\label{github:pygithub3.Github.remaining_requests}\pysigline{\bfcode{remaining\_requests}}
Limit of Github API v3

\end{fulllineitems}

\index{repos (pygithub3.Github attribute)}

\begin{fulllineitems}
\phantomsection\label{github:pygithub3.Github.repos}\pysigline{\bfcode{repos}}
{\hyperref[repos:repos-service]{\emph{Repos service}}}

\end{fulllineitems}

\index{users (pygithub3.Github attribute)}

\begin{fulllineitems}
\phantomsection\label{github:pygithub3.Github.users}\pysigline{\bfcode{users}}
{\hyperref[users:users-service]{\emph{Users service}}}

\end{fulllineitems}


\end{fulllineitems}



\section{Services}
\label{services:services}\label{services::doc}
{\hyperref[github::doc]{\emph{Github}}} class is a glue to all of them and the recommended option to
start


\subsection{Overview}
\label{services:overview}
You can access to the API requests through the different services.

If you take a look at
\href{http://developer.github.com/}{github API v3 documentation}, you'll see a
few sections in the sidebar.

\textbf{pygithub3} has one service per each section of request-related

For example:

\begin{Verbatim}[commandchars=\\\{\}]
repos =\textgreater{} services.repos.repo
    collaborators =\textgreater{} services.repos.collaborators
    commits =\textgreater{} services.repos.commits
    ....
\end{Verbatim}

Each service has the functions to throw the API requests and \textbf{is isolated
from the rest}.


\subsection{Config each service}
\label{services:config-each-service}\label{services:id1}
Each service can be configurated with some variables (behind the scenes, each
service has her client which is configurated with this variables).

\begin{notice}{note}{Note:}
Also you can configure {\hyperref[github::doc]{\emph{Github}}} as a service
\end{notice}
\index{Service (class in pygithub3.services.base)}

\begin{fulllineitems}
\phantomsection\label{services:pygithub3.services.base.Service}\pysiglinewithargsret{\strong{class }\code{pygithub3.services.base.}\bfcode{Service}}{\emph{**config}}{}
You can configure each service with this keyword variables:
\begin{quote}\begin{description}
\item[{Parameters}] \leavevmode\begin{itemize}
\item {} 
\textbf{login} (\emph{str}) -- Username to authenticate

\item {} 
\textbf{password} (\emph{str}) -- Username to authenticate

\item {} 
\textbf{user} (\emph{str}) -- Default username in requests

\item {} 
\textbf{repo} (\emph{str}) -- Default repository in requests

\item {} 
\textbf{token} (\emph{str}) -- Token to OAuth

\item {} 
\textbf{per\_page} (\emph{int}) -- Items in each page of multiple returns

\item {} 
\textbf{base\_url} (\emph{str}) -- To support another github-related API (untested)

\item {} 
\textbf{verbose} (\emph{stream}) -- Stream to write debug logs

\end{itemize}

\end{description}\end{quote}

You can configure the \textbf{authentication} with BasicAuthentication (login
and password) and with \href{http://developer.github.com/v3/oauth/}{OAuth} (
token).
If you include \code{login}, \code{password} and \code{token} in config; Oauth has
precedence

Some API requests need \code{user} and/or \code{repo} arguments (e.g
{\hyperref[repos:config-precedence]{\emph{repos service}}}).
You can configure the default value here to avoid repeating

Some API requests return multiple resources with pagination. You can
configure how many items has each page.

You can configure \code{verbose} logging like \href{http://docs.python-requests.org/en/v0.10.6/user/advanced/\#verbose-logging}{requests library}
\index{set\_credentials() (pygithub3.services.base.Service method)}

\begin{fulllineitems}
\phantomsection\label{services:pygithub3.services.base.Service.set_credentials}\pysiglinewithargsret{\bfcode{set\_credentials}}{\emph{login}, \emph{password}}{}
Set Basic Authentication
\begin{quote}\begin{description}
\item[{Parameters}] \leavevmode\begin{itemize}
\item {} 
\textbf{login} (\emph{str}) -- Username to authenticate

\item {} 
\textbf{password} (\emph{str}) -- Username to authenticate

\end{itemize}

\end{description}\end{quote}

\end{fulllineitems}

\index{set\_repo() (pygithub3.services.base.Service method)}

\begin{fulllineitems}
\phantomsection\label{services:pygithub3.services.base.Service.set_repo}\pysiglinewithargsret{\bfcode{set\_repo}}{\emph{repo}}{}
Set repository
\begin{quote}\begin{description}
\item[{Parameters}] \leavevmode
\textbf{repo} (\emph{str}) -- Default repository in requests

\end{description}\end{quote}

\end{fulllineitems}

\index{set\_token() (pygithub3.services.base.Service method)}

\begin{fulllineitems}
\phantomsection\label{services:pygithub3.services.base.Service.set_token}\pysiglinewithargsret{\bfcode{set\_token}}{\emph{token}}{}
Set OAuth token
\begin{quote}\begin{description}
\item[{Parameters}] \leavevmode
\textbf{token} (\emph{str}) -- Token to OAuth

\end{description}\end{quote}

\end{fulllineitems}

\index{set\_user() (pygithub3.services.base.Service method)}

\begin{fulllineitems}
\phantomsection\label{services:pygithub3.services.base.Service.set_user}\pysiglinewithargsret{\bfcode{set\_user}}{\emph{user}}{}
Set user
\begin{quote}\begin{description}
\item[{Parameters}] \leavevmode
\textbf{user} (\emph{str}) -- Default username in requests

\end{description}\end{quote}

\end{fulllineitems}


\end{fulllineitems}



\subsection{MimeTypes}
\label{services:mimetypes-section}\label{services:mimetypes}
Some services supports \href{http://developer.github.com/v3/mime}{mimetypes}

With them the {\hyperref[resources::doc]{\emph{Resources}}} will have \code{body}, \code{body\_text}, \code{body\_html}
attributes or all of them.
\index{MimeTypeMixin (class in pygithub3.services.base)}

\begin{fulllineitems}
\phantomsection\label{services:pygithub3.services.base.MimeTypeMixin}\pysigline{\strong{class }\code{pygithub3.services.base.}\bfcode{MimeTypeMixin}}
Mimetype support to Services

Adds 4 public functions to service:
\index{set\_full() (pygithub3.services.base.MimeTypeMixin method)}

\begin{fulllineitems}
\phantomsection\label{services:pygithub3.services.base.MimeTypeMixin.set_full}\pysiglinewithargsret{\bfcode{set\_full}}{}{}
Resource will have \code{body}, \code{body\_text} and \code{body\_html}
attributes

\end{fulllineitems}

\index{set\_html() (pygithub3.services.base.MimeTypeMixin method)}

\begin{fulllineitems}
\phantomsection\label{services:pygithub3.services.base.MimeTypeMixin.set_html}\pysiglinewithargsret{\bfcode{set\_html}}{}{}
Resource will have \code{body\_html} attribute

\end{fulllineitems}

\index{set\_raw() (pygithub3.services.base.MimeTypeMixin method)}

\begin{fulllineitems}
\phantomsection\label{services:pygithub3.services.base.MimeTypeMixin.set_raw}\pysiglinewithargsret{\bfcode{set\_raw}}{}{}
Resource will have \code{body} attribute

\end{fulllineitems}

\index{set\_text() (pygithub3.services.base.MimeTypeMixin method)}

\begin{fulllineitems}
\phantomsection\label{services:pygithub3.services.base.MimeTypeMixin.set_text}\pysiglinewithargsret{\bfcode{set\_text}}{}{}
Resource will have \code{body\_text} attribute

\end{fulllineitems}


\end{fulllineitems}


\textbf{Fast example}:

\begin{Verbatim}[commandchars=\\\{\}]
\PYG{k+kn}{from} \PYG{n+nn}{pygithub3} \PYG{k+kn}{import} \PYG{n}{Github}

\PYG{n}{gh} \PYG{o}{=} \PYG{n}{Github}\PYG{p}{(}\PYG{p}{)}

\PYG{n}{gh}\PYG{o}{.}\PYG{n}{gists}\PYG{o}{.}\PYG{n}{comments}\PYG{o}{.}\PYG{n}{set\PYGZus{}html}\PYG{p}{(}\PYG{p}{)}
\PYG{n}{comment} \PYG{o}{=} \PYG{n}{gh}\PYG{o}{.}\PYG{n}{gists}\PYG{o}{.}\PYG{n}{comments}\PYG{o}{.}\PYG{n}{list}\PYG{p}{(}\PYG{l+m+mi}{1}\PYG{p}{)}\PYG{o}{.}\PYG{n}{all}\PYG{p}{(}\PYG{p}{)}\PYG{p}{[}\PYG{l+m+mi}{0}\PYG{p}{]}
\PYG{k}{print} \PYG{n}{comment}\PYG{o}{.}\PYG{n}{body}\PYG{p}{,} \PYG{n}{comment}\PYG{o}{.}\PYG{n}{body\PYGZus{}text}\PYG{p}{,} \PYG{n}{comment}\PYG{o}{.}\PYG{n}{body\PYGZus{}html}
\end{Verbatim}


\subsection{List of services}
\label{services:list-of-services}

\subsubsection{Users services}
\label{users:users-service}\label{users::doc}\label{users:users-services}
\textbf{Fast sample}:

\begin{Verbatim}[commandchars=\\\{\}]
\PYG{k+kn}{from} \PYG{n+nn}{pygithub3} \PYG{k+kn}{import} \PYG{n}{Github}

\PYG{n}{auth} \PYG{o}{=} \PYG{n+nb}{dict}\PYG{p}{(}\PYG{n}{login}\PYG{o}{=}\PYG{l+s}{'}\PYG{l+s}{octocat}\PYG{l+s}{'}\PYG{p}{,} \PYG{n}{password}\PYG{o}{=}\PYG{l+s}{'}\PYG{l+s}{pass}\PYG{l+s}{'}\PYG{p}{)}
\PYG{n}{gh} \PYG{o}{=} \PYG{n}{Github}\PYG{p}{(}\PYG{o}{*}\PYG{o}{*}\PYG{n}{auth}\PYG{p}{)}

\PYG{c}{\PYGZsh{} Get copitux user}
\PYG{n}{gh}\PYG{o}{.}\PYG{n}{users}\PYG{o}{.}\PYG{n}{get}\PYG{p}{(}\PYG{l+s}{'}\PYG{l+s}{copitux}\PYG{l+s}{'}\PYG{p}{)}

\PYG{c}{\PYGZsh{} Get copitux's followers}
\PYG{n}{gh}\PYG{o}{.}\PYG{n}{users}\PYG{o}{.}\PYG{n}{followers}\PYG{o}{.}\PYG{n}{list}\PYG{p}{(}\PYG{l+s}{'}\PYG{l+s}{copitux}\PYG{l+s}{'}\PYG{p}{)}

\PYG{c}{\PYGZsh{} Get octocat's emails}
\PYG{n}{gh}\PYG{o}{.}\PYG{n}{users}\PYG{o}{.}\PYG{n}{emails}\PYG{o}{.}\PYG{n}{list}\PYG{p}{(}\PYG{p}{)}
\end{Verbatim}


\paragraph{User}
\label{users:user}\index{User (class in pygithub3.services.users)}

\begin{fulllineitems}
\phantomsection\label{users:pygithub3.services.users.User}\pysiglinewithargsret{\strong{class }\code{pygithub3.services.users.}\bfcode{User}}{\emph{**config}}{}
Consume \href{http://developer.github.com/v3/users}{Users API}
\index{emails (User attribute)}

\begin{fulllineitems}
\phantomsection\label{users:User.emails}\pysigline{\bfcode{emails}}
{\hyperref[users:emails-service]{\emph{Emails}}}

\end{fulllineitems}

\index{keys (User attribute)}

\begin{fulllineitems}
\phantomsection\label{users:User.keys}\pysigline{\bfcode{keys}}
{\hyperref[users:keys-service]{\emph{Keys}}}

\end{fulllineitems}

\index{followers (User attribute)}

\begin{fulllineitems}
\phantomsection\label{users:User.followers}\pysigline{\bfcode{followers}}
{\hyperref[users:followers-service]{\emph{Followers}}}

\end{fulllineitems}

\index{get() (pygithub3.services.users.User method)}

\begin{fulllineitems}
\phantomsection\label{users:pygithub3.services.users.User.get}\pysiglinewithargsret{\bfcode{get}}{\emph{user=None}}{}
Get a single user
\begin{quote}\begin{description}
\item[{Parameters}] \leavevmode
\textbf{user} (\emph{str}) -- Username

\end{description}\end{quote}

If you call it without user and you are authenticated, get the
authenticated user.

\begin{notice}{warning}{Warning:}
If you aren't authenticated and call without user, it returns 403
\end{notice}

\begin{Verbatim}[commandchars=\\\{\}]
\PYG{n}{user\PYGZus{}service}\PYG{o}{.}\PYG{n}{get}\PYG{p}{(}\PYG{l+s}{'}\PYG{l+s}{copitux}\PYG{l+s}{'}\PYG{p}{)}
\PYG{n}{user\PYGZus{}service}\PYG{o}{.}\PYG{n}{get}\PYG{p}{(}\PYG{p}{)}
\end{Verbatim}

\end{fulllineitems}

\index{update() (pygithub3.services.users.User method)}

\begin{fulllineitems}
\phantomsection\label{users:pygithub3.services.users.User.update}\pysiglinewithargsret{\bfcode{update}}{\emph{data}}{}
Update the authenticated user
\begin{quote}\begin{description}
\item[{Parameters}] \leavevmode
\textbf{data} (\emph{dict}) -- Input to update

\end{description}\end{quote}

\begin{Verbatim}[commandchars=\\\{\}]
\PYG{n}{user\PYGZus{}service}\PYG{o}{.}\PYG{n}{update}\PYG{p}{(}\PYG{n+nb}{dict}\PYG{p}{(}\PYG{n}{name}\PYG{o}{=}\PYG{l+s}{'}\PYG{l+s}{new\PYGZus{}name}\PYG{l+s}{'}\PYG{p}{,} \PYG{n}{bio}\PYG{o}{=}\PYG{l+s}{'}\PYG{l+s}{new\PYGZus{}bio}\PYG{l+s}{'}\PYG{p}{)}\PYG{p}{)}
\end{Verbatim}

\end{fulllineitems}


\end{fulllineitems}



\paragraph{Emails}
\label{users:emails-service}\label{users:emails}\index{Emails (class in pygithub3.services.users)}

\begin{fulllineitems}
\phantomsection\label{users:pygithub3.services.users.Emails}\pysiglinewithargsret{\strong{class }\code{pygithub3.services.users.}\bfcode{Emails}}{\emph{**config}}{}
Consume \href{http://developer.github.com/v3/users/emails/}{Emails API}

\begin{notice}{warning}{Warning:}
You must be authenticated for all requests
\end{notice}
\index{add() (pygithub3.services.users.Emails method)}

\begin{fulllineitems}
\phantomsection\label{users:pygithub3.services.users.Emails.add}\pysiglinewithargsret{\bfcode{add}}{\emph{*emails}}{}
Add emails
\begin{quote}\begin{description}
\item[{Parameters}] \leavevmode
\textbf{emails} (\emph{list}) -- Emails to add

\end{description}\end{quote}

\begin{notice}{note}{Note:}
It rejects non-valid emails
\end{notice}

\begin{Verbatim}[commandchars=\\\{\}]
\PYG{n}{email\PYGZus{}service}\PYG{o}{.}\PYG{n}{add}\PYG{p}{(}\PYG{l+s}{'}\PYG{l+s}{test1@xample.com}\PYG{l+s}{'}\PYG{p}{,} \PYG{l+s}{'}\PYG{l+s}{test2@xample.com}\PYG{l+s}{'}\PYG{p}{)}
\end{Verbatim}

\end{fulllineitems}

\index{delete() (pygithub3.services.users.Emails method)}

\begin{fulllineitems}
\phantomsection\label{users:pygithub3.services.users.Emails.delete}\pysiglinewithargsret{\bfcode{delete}}{\emph{*emails}}{}
Delete emails
\begin{quote}\begin{description}
\item[{Parameters}] \leavevmode
\textbf{emails} (\emph{list}) -- List of emails

\end{description}\end{quote}

\begin{Verbatim}[commandchars=\\\{\}]
\PYG{n}{email\PYGZus{}service}\PYG{o}{.}\PYG{n}{delete}\PYG{p}{(}\PYG{l+s}{'}\PYG{l+s}{test1@xample.com}\PYG{l+s}{'}\PYG{p}{,} \PYG{l+s}{'}\PYG{l+s}{test2@xample.com}\PYG{l+s}{'}\PYG{p}{)}
\end{Verbatim}

\end{fulllineitems}

\index{list() (pygithub3.services.users.Emails method)}

\begin{fulllineitems}
\phantomsection\label{users:pygithub3.services.users.Emails.list}\pysiglinewithargsret{\bfcode{list}}{}{}
Get user's emails
\begin{quote}\begin{description}
\item[{Returns}] \leavevmode
A {\hyperref[result::doc]{\emph{Result}}}

\end{description}\end{quote}

\end{fulllineitems}


\end{fulllineitems}



\paragraph{Keys}
\label{users:keys}\label{users:keys-service}\index{Keys (class in pygithub3.services.users)}

\begin{fulllineitems}
\phantomsection\label{users:pygithub3.services.users.Keys}\pysiglinewithargsret{\strong{class }\code{pygithub3.services.users.}\bfcode{Keys}}{\emph{**config}}{}
Consume \href{http://developer.github.com/v3/users/keys/}{Keys API}

\begin{notice}{warning}{Warning:}
You must be authenticated for all requests
\end{notice}
\index{add() (pygithub3.services.users.Keys method)}

\begin{fulllineitems}
\phantomsection\label{users:pygithub3.services.users.Keys.add}\pysiglinewithargsret{\bfcode{add}}{\emph{data}}{}
Add a public key
\begin{quote}\begin{description}
\item[{Parameters}] \leavevmode
\textbf{data} (\emph{dict}) -- Key (title and key attributes required)

\end{description}\end{quote}

\begin{Verbatim}[commandchars=\\\{\}]
\PYG{n}{key\PYGZus{}service}\PYG{o}{.}\PYG{n}{add}\PYG{p}{(}\PYG{n+nb}{dict}\PYG{p}{(}\PYG{n}{title}\PYG{o}{=}\PYG{l+s}{'}\PYG{l+s}{host}\PYG{l+s}{'}\PYG{p}{,} \PYG{n}{key}\PYG{o}{=}\PYG{l+s}{'}\PYG{l+s}{ssh-rsa AAA...}\PYG{l+s}{'}\PYG{p}{)}\PYG{p}{)}
\end{Verbatim}

\end{fulllineitems}

\index{delete() (pygithub3.services.users.Keys method)}

\begin{fulllineitems}
\phantomsection\label{users:pygithub3.services.users.Keys.delete}\pysiglinewithargsret{\bfcode{delete}}{\emph{key\_id}}{}
Delete a public key
\begin{quote}\begin{description}
\item[{Parameters}] \leavevmode
\textbf{key\_id} (\emph{int}) -- Key id

\end{description}\end{quote}

\end{fulllineitems}

\index{get() (pygithub3.services.users.Keys method)}

\begin{fulllineitems}
\phantomsection\label{users:pygithub3.services.users.Keys.get}\pysiglinewithargsret{\bfcode{get}}{\emph{key\_id}}{}
Get a public key
\begin{quote}\begin{description}
\item[{Parameters}] \leavevmode
\textbf{key\_id} (\emph{int}) -- Key id

\end{description}\end{quote}

\end{fulllineitems}

\index{list() (pygithub3.services.users.Keys method)}

\begin{fulllineitems}
\phantomsection\label{users:pygithub3.services.users.Keys.list}\pysiglinewithargsret{\bfcode{list}}{}{}
Get public keys
\begin{quote}\begin{description}
\item[{Returns}] \leavevmode
A {\hyperref[result::doc]{\emph{Result}}}

\end{description}\end{quote}

\end{fulllineitems}

\index{update() (pygithub3.services.users.Keys method)}

\begin{fulllineitems}
\phantomsection\label{users:pygithub3.services.users.Keys.update}\pysiglinewithargsret{\bfcode{update}}{\emph{key\_id}, \emph{data}}{}
Update a public key
\begin{quote}\begin{description}
\item[{Parameters}] \leavevmode\begin{itemize}
\item {} 
\textbf{key\_id} (\emph{int}) -- Key id

\item {} 
\textbf{data} (\emph{dict}) -- Key (title and key attributes required)

\end{itemize}

\end{description}\end{quote}

\begin{Verbatim}[commandchars=\\\{\}]
\PYG{n}{key\PYGZus{}service}\PYG{o}{.}\PYG{n}{update}\PYG{p}{(}\PYG{l+m+mi}{42}\PYG{p}{,} \PYG{n+nb}{dict}\PYG{p}{(}\PYG{n}{title}\PYG{o}{=}\PYG{l+s}{'}\PYG{l+s}{host}\PYG{l+s}{'}\PYG{p}{,} \PYG{n}{key}\PYG{o}{=}\PYG{l+s}{'}\PYG{l+s}{ssh-rsa AAA...}\PYG{l+s}{'}\PYG{p}{)}\PYG{p}{)}
\end{Verbatim}

\end{fulllineitems}


\end{fulllineitems}



\paragraph{Followers}
\label{users:followers}\label{users:followers-service}\index{Followers (class in pygithub3.services.users)}

\begin{fulllineitems}
\phantomsection\label{users:pygithub3.services.users.Followers}\pysiglinewithargsret{\strong{class }\code{pygithub3.services.users.}\bfcode{Followers}}{\emph{**config}}{}
Consume \href{http://developer.github.com/v3/users/followers/}{Followers API}
\index{follow() (pygithub3.services.users.Followers method)}

\begin{fulllineitems}
\phantomsection\label{users:pygithub3.services.users.Followers.follow}\pysiglinewithargsret{\bfcode{follow}}{\emph{user}}{}
Follow a user
\begin{quote}\begin{description}
\item[{Parameters}] \leavevmode
\textbf{user} (\emph{str}) -- Username

\end{description}\end{quote}

\begin{notice}{warning}{Warning:}
You must be authenticated
\end{notice}

\end{fulllineitems}

\index{is\_following() (pygithub3.services.users.Followers method)}

\begin{fulllineitems}
\phantomsection\label{users:pygithub3.services.users.Followers.is_following}\pysiglinewithargsret{\bfcode{is\_following}}{\emph{user}}{}
Check if you are following a user
\begin{quote}\begin{description}
\item[{Parameters}] \leavevmode
\textbf{user} (\emph{str}) -- Username

\end{description}\end{quote}

\end{fulllineitems}

\index{list() (pygithub3.services.users.Followers method)}

\begin{fulllineitems}
\phantomsection\label{users:pygithub3.services.users.Followers.list}\pysiglinewithargsret{\bfcode{list}}{\emph{user=None}}{}
Get user's followers
\begin{quote}\begin{description}
\item[{Parameters}] \leavevmode
\textbf{user} (\emph{str}) -- Username

\item[{Returns}] \leavevmode
A {\hyperref[result::doc]{\emph{Result}}}

\end{description}\end{quote}

If you call it without user and you are authenticated, get the
authenticated user's followers

\begin{notice}{warning}{Warning:}
If you aren't authenticated and call without user, it returns 403
\end{notice}

\begin{Verbatim}[commandchars=\\\{\}]
\PYG{n}{followers\PYGZus{}service}\PYG{o}{.}\PYG{n}{list}\PYG{p}{(}\PYG{p}{)}
\PYG{n}{followers\PYGZus{}service}\PYG{o}{.}\PYG{n}{list}\PYG{p}{(}\PYG{l+s}{'}\PYG{l+s}{octocat}\PYG{l+s}{'}\PYG{p}{)}
\end{Verbatim}

\end{fulllineitems}

\index{list\_following() (pygithub3.services.users.Followers method)}

\begin{fulllineitems}
\phantomsection\label{users:pygithub3.services.users.Followers.list_following}\pysiglinewithargsret{\bfcode{list\_following}}{\emph{user=None}}{}
Get who a user is following
\begin{quote}\begin{description}
\item[{Parameters}] \leavevmode
\textbf{user} (\emph{str}) -- Username

\item[{Returns}] \leavevmode
A {\hyperref[result::doc]{\emph{Result}}}

\end{description}\end{quote}

If you call it without user and you are authenticated, get the
authenticated user's followings

\begin{notice}{warning}{Warning:}
If you aren't authenticated and call without user, it returns 403
\end{notice}

\begin{Verbatim}[commandchars=\\\{\}]
\PYG{n}{followers\PYGZus{}service}\PYG{o}{.}\PYG{n}{list\PYGZus{}following}\PYG{p}{(}\PYG{p}{)}
\PYG{n}{followers\PYGZus{}service}\PYG{o}{.}\PYG{n}{list\PYGZus{}following}\PYG{p}{(}\PYG{l+s}{'}\PYG{l+s}{octocat}\PYG{l+s}{'}\PYG{p}{)}
\end{Verbatim}

\end{fulllineitems}

\index{unfollow() (pygithub3.services.users.Followers method)}

\begin{fulllineitems}
\phantomsection\label{users:pygithub3.services.users.Followers.unfollow}\pysiglinewithargsret{\bfcode{unfollow}}{\emph{user}}{}
Unfollow a user
\begin{quote}\begin{description}
\item[{Parameters}] \leavevmode
\textbf{user} (\emph{str}) -- Username

\end{description}\end{quote}

\begin{notice}{warning}{Warning:}
You must be authenticated
\end{notice}

\end{fulllineitems}


\end{fulllineitems}



\subsubsection{Repos services}
\label{repos:repos-service}\label{repos::doc}\label{repos:repos-services}
\textbf{Fast sample}:

\begin{Verbatim}[commandchars=\\\{\}]
\PYG{k+kn}{from} \PYG{n+nn}{pygithub3} \PYG{k+kn}{import} \PYG{n}{Github}

\PYG{n}{gh} \PYG{o}{=} \PYG{n}{Github}\PYG{p}{(}\PYG{p}{)}

\PYG{n}{django\PYGZus{}compressor} \PYG{o}{=} \PYG{n}{gh}\PYG{o}{.}\PYG{n}{repos}\PYG{o}{.}\PYG{n}{get}\PYG{p}{(}\PYG{n}{user}\PYG{o}{=}\PYG{l+s}{'}\PYG{l+s}{jezdez}\PYG{l+s}{'}\PYG{p}{,} \PYG{n}{repo}\PYG{o}{=}\PYG{l+s}{'}\PYG{l+s}{django\PYGZus{}compressor}\PYG{l+s}{'}\PYG{p}{)}
\PYG{n}{requests\PYGZus{}collaborators} \PYG{o}{=} \PYG{n}{gh}\PYG{o}{.}\PYG{n}{repos}\PYG{o}{.}\PYG{n}{collaborators}\PYG{p}{(}\PYG{n}{user}\PYG{o}{=}\PYG{l+s}{'}\PYG{l+s}{kennethreitz}\PYG{l+s}{'}\PYG{p}{,}
    \PYG{n}{repo}\PYG{o}{=}\PYG{l+s}{'}\PYG{l+s}{requests}\PYG{l+s}{'}\PYG{p}{)}
\end{Verbatim}


\paragraph{Config precedence}
\label{repos:config-precedence}\label{repos:id1}
Some request always need \code{user} and \code{repo} parameters, both, to identify
a \emph{repository}. Because there are a lot of requests which need that parameters,
you can {\hyperref[services:config-each-service]{\emph{Config each service}}} with \code{user} and \code{repo} globally.

So several requests follow a simple precedence
\code{user\_in\_arg \textgreater{} user\_in\_config \textbar{} repo\_in\_arg \textgreater{} repo\_in\_config}

You can see it better with an example:

\begin{Verbatim}[commandchars=\\\{\}]
\PYG{k+kn}{from} \PYG{n+nn}{pygithub3} \PYG{k+kn}{import} \PYG{n}{Github}

\PYG{n}{gh} \PYG{o}{=} \PYG{n}{Github}\PYG{p}{(}\PYG{n}{user}\PYG{o}{=}\PYG{l+s}{'}\PYG{l+s}{octocat}\PYG{l+s}{'}\PYG{p}{,} \PYG{n}{repo}\PYG{o}{=}\PYG{l+s}{'}\PYG{l+s}{oct\PYGZus{}repo}\PYG{l+s}{'}\PYG{p}{)}
\PYG{n}{oct\PYGZus{}repo} \PYG{o}{=} \PYG{n}{gh}\PYG{o}{.}\PYG{n}{repos}\PYG{o}{.}\PYG{n}{get}\PYG{p}{(}\PYG{p}{)}
\PYG{n}{another\PYGZus{}repo\PYGZus{}from\PYGZus{}octocat} \PYG{o}{=} \PYG{n}{gh}\PYG{o}{.}\PYG{n}{repos}\PYG{o}{.}\PYG{n}{get}\PYG{p}{(}\PYG{n}{repo}\PYG{o}{=}\PYG{l+s}{'}\PYG{l+s}{another\PYGZus{}repo}\PYG{l+s}{'}\PYG{p}{)}

\PYG{n}{django\PYGZus{}compressor} \PYG{o}{=} \PYG{n}{gh}\PYG{o}{.}\PYG{n}{repos}\PYG{o}{.}\PYG{n}{get}\PYG{p}{(}\PYG{n}{user}\PYG{o}{=}\PYG{l+s}{'}\PYG{l+s}{jezdez}\PYG{l+s}{'}\PYG{p}{,} \PYG{n}{repo}\PYG{o}{=}\PYG{l+s}{'}\PYG{l+s}{django\PYGZus{}compressor}\PYG{l+s}{'}\PYG{p}{)}
\end{Verbatim}

\begin{notice}{note}{Note:}
Remember that each service is isolated from the rest

\begin{Verbatim}[commandchars=\\\{\}]
\PYG{c}{\PYGZsh{} continue example...}
\PYG{n}{gh}\PYG{o}{.}\PYG{n}{repos}\PYG{o}{.}\PYG{n}{commits}\PYG{o}{.}\PYG{n}{set\PYGZus{}user}\PYG{p}{(}\PYG{l+s}{'}\PYG{l+s}{copitux}\PYG{l+s}{'}\PYG{p}{)}
\PYG{n}{oct\PYGZus{}repo} \PYG{o}{=} \PYG{n}{gh}\PYG{o}{.}\PYG{n}{repos}\PYG{o}{.}\PYG{n}{get}\PYG{p}{(}\PYG{p}{)}
\PYG{n}{oct\PYGZus{}repo\PYGZus{}collaborators} \PYG{o}{=} \PYG{n}{gh}\PYG{o}{.}\PYG{n}{repos}\PYG{o}{.}\PYG{n}{collaborators}\PYG{o}{.}\PYG{n}{list}\PYG{p}{(}\PYG{p}{)}\PYG{o}{.}\PYG{n}{all}\PYG{p}{(}\PYG{p}{)}

\PYG{c}{\PYGZsh{} Fail because copitux/oct\PYGZus{}repo doesn't exist}
\PYG{n}{gh}\PYG{o}{.}\PYG{n}{repos}\PYG{o}{.}\PYG{n}{commits}\PYG{o}{.}\PYG{n}{list\PYGZus{}comments}\PYG{p}{(}\PYG{p}{)}
\end{Verbatim}
\end{notice}


\paragraph{Repo}
\label{repos:repo}\index{Repo (class in pygithub3.services.repos)}

\begin{fulllineitems}
\phantomsection\label{repos:pygithub3.services.repos.Repo}\pysiglinewithargsret{\strong{class }\code{pygithub3.services.repos.}\bfcode{Repo}}{\emph{**config}}{}
Consume \href{http://developer.github.com/v3/repos}{Repos API}
\index{collaborators (Repo attribute)}

\begin{fulllineitems}
\phantomsection\label{repos:Repo.collaborators}\pysigline{\bfcode{collaborators}}
{\hyperref[repos:collaborators-service]{\emph{Collaborators}}}

\end{fulllineitems}

\index{commits (Repo attribute)}

\begin{fulllineitems}
\phantomsection\label{repos:Repo.commits}\pysigline{\bfcode{commits}}
{\hyperref[repos:commits-service]{\emph{Commits}}}

\end{fulllineitems}

\index{downloads (Repo attribute)}

\begin{fulllineitems}
\phantomsection\label{repos:Repo.downloads}\pysigline{\bfcode{downloads}}
{\hyperref[repos:downloads-service]{\emph{Downloads}}}

\end{fulllineitems}

\index{forks (Repo attribute)}

\begin{fulllineitems}
\phantomsection\label{repos:Repo.forks}\pysigline{\bfcode{forks}}
{\hyperref[repos:forks-service]{\emph{Forks}}}

\end{fulllineitems}

\index{keys (Repo attribute)}

\begin{fulllineitems}
\phantomsection\label{repos:Repo.keys}\pysigline{\bfcode{keys}}
{\hyperref[repos:repokeys-service]{\emph{Keys}}}

\end{fulllineitems}

\index{watchers (Repo attribute)}

\begin{fulllineitems}
\phantomsection\label{repos:Repo.watchers}\pysigline{\bfcode{watchers}}
{\hyperref[repos:watchers-service]{\emph{Watchers}}}

\end{fulllineitems}

\index{create() (pygithub3.services.repos.Repo method)}

\begin{fulllineitems}
\phantomsection\label{repos:pygithub3.services.repos.Repo.create}\pysiglinewithargsret{\bfcode{create}}{\emph{data}, \emph{in\_org=None}}{}
Create a new repository
\begin{quote}\begin{description}
\item[{Parameters}] \leavevmode\begin{itemize}
\item {} 
\textbf{data} (\emph{dict}) -- Input. See \href{http://developer.github.com/v3/repos}{github repos doc}

\item {} 
\textbf{in\_org} (\emph{str}) -- Organization where create the repository (optional)

\end{itemize}

\end{description}\end{quote}

\begin{notice}{warning}{Warning:}
You must be authenticated

If you use \code{in\_org} arg, the authenticated user must be a member
of \textless{}in\_org\textgreater{}
\end{notice}

\begin{Verbatim}[commandchars=\\\{\}]
\PYG{n}{repo\PYGZus{}service}\PYG{o}{.}\PYG{n}{create}\PYG{p}{(}\PYG{n+nb}{dict}\PYG{p}{(}\PYG{n}{name}\PYG{o}{=}\PYG{l+s}{'}\PYG{l+s}{new\PYGZus{}repo}\PYG{l+s}{'}\PYG{p}{,} \PYG{n}{description}\PYG{o}{=}\PYG{l+s}{'}\PYG{l+s}{desc}\PYG{l+s}{'}\PYG{p}{)}\PYG{p}{)}
\PYG{n}{repo\PYGZus{}service}\PYG{o}{.}\PYG{n}{create}\PYG{p}{(}\PYG{n+nb}{dict}\PYG{p}{(}\PYG{n}{name}\PYG{o}{=}\PYG{l+s}{'}\PYG{l+s}{new\PYGZus{}repo\PYGZus{}in\PYGZus{}org}\PYG{l+s}{'}\PYG{p}{,} \PYG{n}{team\PYGZus{}id}\PYG{o}{=}\PYG{l+m+mi}{2300}\PYG{p}{)}\PYG{p}{,}
    \PYG{n}{in\PYGZus{}org}\PYG{o}{=}\PYG{l+s}{'}\PYG{l+s}{myorganization}\PYG{l+s}{'}\PYG{p}{)}
\end{Verbatim}

\end{fulllineitems}

\index{get() (pygithub3.services.repos.Repo method)}

\begin{fulllineitems}
\phantomsection\label{repos:pygithub3.services.repos.Repo.get}\pysiglinewithargsret{\bfcode{get}}{\emph{user=None}, \emph{repo=None}}{}
Get a single repo
\begin{quote}\begin{description}
\item[{Parameters}] \leavevmode\begin{itemize}
\item {} 
\textbf{user} (\emph{str}) -- Username

\item {} 
\textbf{repo} (\emph{str}) -- Repository

\end{itemize}

\end{description}\end{quote}

\begin{notice}{note}{Note:}
Remember {\hyperref[repos:config-precedence]{\emph{Config precedence}}}
\end{notice}

\end{fulllineitems}

\index{list() (pygithub3.services.repos.Repo method)}

\begin{fulllineitems}
\phantomsection\label{repos:pygithub3.services.repos.Repo.list}\pysiglinewithargsret{\bfcode{list}}{\emph{user=None}, \emph{type='all'}}{}
Get user's repositories
\begin{quote}\begin{description}
\item[{Parameters}] \leavevmode\begin{itemize}
\item {} 
\textbf{user} (\emph{str}) -- Username

\item {} 
\textbf{type} (\emph{str}) -- Filter by type (optional). See \href{http://developer.github.com/v3/repos}{github repos doc}

\end{itemize}

\item[{Returns}] \leavevmode
A {\hyperref[result::doc]{\emph{Result}}}

\end{description}\end{quote}

If you call it without user and you are authenticated, get the
authenticated user's repositories

\begin{notice}{warning}{Warning:}
If you aren't authenticated and call without user, it returns 403
\end{notice}

\begin{Verbatim}[commandchars=\\\{\}]
\PYG{n}{repo\PYGZus{}service}\PYG{o}{.}\PYG{n}{list}\PYG{p}{(}\PYG{l+s}{'}\PYG{l+s}{copitux}\PYG{l+s}{'}\PYG{p}{,} \PYG{n+nb}{type}\PYG{o}{=}\PYG{l+s}{'}\PYG{l+s}{owner}\PYG{l+s}{'}\PYG{p}{)}
\PYG{n}{repo\PYGZus{}service}\PYG{o}{.}\PYG{n}{list}\PYG{p}{(}\PYG{n+nb}{type}\PYG{o}{=}\PYG{l+s}{'}\PYG{l+s}{private}\PYG{l+s}{'}\PYG{p}{)}
\end{Verbatim}

\end{fulllineitems}

\index{list\_branches() (pygithub3.services.repos.Repo method)}

\begin{fulllineitems}
\phantomsection\label{repos:pygithub3.services.repos.Repo.list_branches}\pysiglinewithargsret{\bfcode{list\_branches}}{\emph{user=None}, \emph{repo=None}}{}
Get repository's branches
\begin{quote}\begin{description}
\item[{Parameters}] \leavevmode\begin{itemize}
\item {} 
\textbf{user} (\emph{str}) -- Username

\item {} 
\textbf{repo} (\emph{str}) -- Repository

\end{itemize}

\item[{Returns}] \leavevmode
A {\hyperref[result::doc]{\emph{Result}}}

\end{description}\end{quote}

\begin{notice}{note}{Note:}
Remember {\hyperref[repos:config-precedence]{\emph{Config precedence}}}
\end{notice}

\end{fulllineitems}

\index{list\_by\_org() (pygithub3.services.repos.Repo method)}

\begin{fulllineitems}
\phantomsection\label{repos:pygithub3.services.repos.Repo.list_by_org}\pysiglinewithargsret{\bfcode{list\_by\_org}}{\emph{org}, \emph{type='all'}}{}
Get organization's repositories
\begin{quote}\begin{description}
\item[{Parameters}] \leavevmode\begin{itemize}
\item {} 
\textbf{org} (\emph{str}) -- Organization name

\item {} 
\textbf{type} (\emph{str}) -- Filter by type (optional). See \href{http://developer.github.com/v3/repos}{github repos doc}

\end{itemize}

\item[{Returns}] \leavevmode
A {\hyperref[result::doc]{\emph{Result}}}

\end{description}\end{quote}

\begin{Verbatim}[commandchars=\\\{\}]
\PYG{n}{repo\PYGZus{}service}\PYG{o}{.}\PYG{n}{list\PYGZus{}by\PYGZus{}org}\PYG{p}{(}\PYG{l+s}{'}\PYG{l+s}{myorganization}\PYG{l+s}{'}\PYG{p}{,} \PYG{n+nb}{type}\PYG{o}{=}\PYG{l+s}{'}\PYG{l+s}{member}\PYG{l+s}{'}\PYG{p}{)}
\end{Verbatim}

\end{fulllineitems}

\index{list\_contributors() (pygithub3.services.repos.Repo method)}

\begin{fulllineitems}
\phantomsection\label{repos:pygithub3.services.repos.Repo.list_contributors}\pysiglinewithargsret{\bfcode{list\_contributors}}{\emph{user=None}, \emph{repo=None}}{}
Get repository's contributors
\begin{quote}\begin{description}
\item[{Parameters}] \leavevmode\begin{itemize}
\item {} 
\textbf{user} (\emph{str}) -- Username

\item {} 
\textbf{repo} (\emph{str}) -- Repository

\end{itemize}

\item[{Returns}] \leavevmode
A {\hyperref[result::doc]{\emph{Result}}}

\end{description}\end{quote}

\begin{notice}{note}{Note:}
Remember {\hyperref[repos:config-precedence]{\emph{Config precedence}}}
\end{notice}

\end{fulllineitems}

\index{list\_contributors\_with\_anonymous() (pygithub3.services.repos.Repo method)}

\begin{fulllineitems}
\phantomsection\label{repos:pygithub3.services.repos.Repo.list_contributors_with_anonymous}\pysiglinewithargsret{\bfcode{list\_contributors\_with\_anonymous}}{\emph{user=None}, \emph{repo=None}}{}
Like {\hyperref[repos:pygithub3.services.repos.Repo.list_contributors]{\code{list\_contributors}}} plus
anonymous

\end{fulllineitems}

\index{list\_languages() (pygithub3.services.repos.Repo method)}

\begin{fulllineitems}
\phantomsection\label{repos:pygithub3.services.repos.Repo.list_languages}\pysiglinewithargsret{\bfcode{list\_languages}}{\emph{user=None}, \emph{repo=None}}{}
Get repository's languages
\begin{quote}\begin{description}
\item[{Parameters}] \leavevmode\begin{itemize}
\item {} 
\textbf{user} (\emph{str}) -- Username

\item {} 
\textbf{repo} (\emph{str}) -- Repository

\end{itemize}

\item[{Returns}] \leavevmode
A {\hyperref[result::doc]{\emph{Result}}}

\end{description}\end{quote}

\begin{notice}{note}{Note:}
Remember {\hyperref[repos:config-precedence]{\emph{Config precedence}}}
\end{notice}

\end{fulllineitems}

\index{list\_tags() (pygithub3.services.repos.Repo method)}

\begin{fulllineitems}
\phantomsection\label{repos:pygithub3.services.repos.Repo.list_tags}\pysiglinewithargsret{\bfcode{list\_tags}}{\emph{user=None}, \emph{repo=None}}{}
Get repository's tags
\begin{quote}\begin{description}
\item[{Parameters}] \leavevmode\begin{itemize}
\item {} 
\textbf{user} (\emph{str}) -- Username

\item {} 
\textbf{repo} (\emph{str}) -- Repository

\end{itemize}

\item[{Returns}] \leavevmode
A {\hyperref[result::doc]{\emph{Result}}}

\end{description}\end{quote}

\begin{notice}{note}{Note:}
Remember {\hyperref[repos:config-precedence]{\emph{Config precedence}}}
\end{notice}

\end{fulllineitems}

\index{list\_teams() (pygithub3.services.repos.Repo method)}

\begin{fulllineitems}
\phantomsection\label{repos:pygithub3.services.repos.Repo.list_teams}\pysiglinewithargsret{\bfcode{list\_teams}}{\emph{user=None}, \emph{repo=None}}{}
Get repository's teams
\begin{quote}\begin{description}
\item[{Parameters}] \leavevmode\begin{itemize}
\item {} 
\textbf{user} (\emph{str}) -- Username

\item {} 
\textbf{repo} (\emph{str}) -- Repository

\end{itemize}

\item[{Returns}] \leavevmode
A {\hyperref[result::doc]{\emph{Result}}}

\end{description}\end{quote}

\begin{notice}{note}{Note:}
Remember {\hyperref[repos:config-precedence]{\emph{Config precedence}}}
\end{notice}

\end{fulllineitems}

\index{update() (pygithub3.services.repos.Repo method)}

\begin{fulllineitems}
\phantomsection\label{repos:pygithub3.services.repos.Repo.update}\pysiglinewithargsret{\bfcode{update}}{\emph{data}, \emph{user=None}, \emph{repo=None}}{}
Update a single repository
\begin{quote}\begin{description}
\item[{Parameters}] \leavevmode\begin{itemize}
\item {} 
\textbf{data} (\emph{dict}) -- Input. See \href{http://developer.github.com/v3/repos}{github repos doc}

\item {} 
\textbf{user} (\emph{str}) -- Username

\item {} 
\textbf{repo} (\emph{str}) -- Repository

\end{itemize}

\end{description}\end{quote}

\begin{notice}{note}{Note:}
Remember {\hyperref[repos:config-precedence]{\emph{Config precedence}}}
\end{notice}

\begin{notice}{warning}{Warning:}
You must be authenticated
\end{notice}

\begin{Verbatim}[commandchars=\\\{\}]
\PYG{n}{repo\PYGZus{}service}\PYG{o}{.}\PYG{n}{update}\PYG{p}{(}\PYG{n+nb}{dict}\PYG{p}{(}\PYG{n}{has\PYGZus{}issues}\PYG{o}{=}\PYG{n+nb+bp}{True}\PYG{p}{)}\PYG{p}{,} \PYG{n}{user}\PYG{o}{=}\PYG{l+s}{'}\PYG{l+s}{octocat}\PYG{l+s}{'}\PYG{p}{,}
    \PYG{n}{repo}\PYG{o}{=}\PYG{l+s}{'}\PYG{l+s}{oct\PYGZus{}repo}\PYG{l+s}{'}\PYG{p}{)}
\end{Verbatim}

\end{fulllineitems}


\end{fulllineitems}



\paragraph{Collaborators}
\label{repos:collaborators}\label{repos:collaborators-service}\index{Collaborators (class in pygithub3.services.repos)}

\begin{fulllineitems}
\phantomsection\label{repos:pygithub3.services.repos.Collaborators}\pysiglinewithargsret{\strong{class }\code{pygithub3.services.repos.}\bfcode{Collaborators}}{\emph{**config}}{}
Consume \href{http://developer.github.com/v3/repos/collaborators}{Repo Collaborators API}
\index{add() (pygithub3.services.repos.Collaborators method)}

\begin{fulllineitems}
\phantomsection\label{repos:pygithub3.services.repos.Collaborators.add}\pysiglinewithargsret{\bfcode{add}}{\emph{collaborator}, \emph{user=None}, \emph{repo=None}}{}
Add collaborator to a repository
\begin{quote}\begin{description}
\item[{Parameters}] \leavevmode\begin{itemize}
\item {} 
\textbf{collaborator} (\emph{str}) -- Collaborator's username

\item {} 
\textbf{user} (\emph{str}) -- Username

\item {} 
\textbf{repo} (\emph{str}) -- Repository

\end{itemize}

\end{description}\end{quote}

\begin{notice}{note}{Note:}
Remember {\hyperref[repos:config-precedence]{\emph{Config precedence}}}
\end{notice}

\begin{notice}{warning}{Warning:}
You must be authenticated and have perms in repository
\end{notice}

\end{fulllineitems}

\index{delete() (pygithub3.services.repos.Collaborators method)}

\begin{fulllineitems}
\phantomsection\label{repos:pygithub3.services.repos.Collaborators.delete}\pysiglinewithargsret{\bfcode{delete}}{\emph{collaborator}, \emph{user=None}, \emph{repo=None}}{}
Remove collaborator from repository
\begin{quote}\begin{description}
\item[{Parameters}] \leavevmode\begin{itemize}
\item {} 
\textbf{collaborator} (\emph{str}) -- Collaborator's username

\item {} 
\textbf{user} (\emph{str}) -- Username

\item {} 
\textbf{repo} (\emph{str}) -- Repository

\end{itemize}

\end{description}\end{quote}

\begin{notice}{note}{Note:}
Remember {\hyperref[repos:config-precedence]{\emph{Config precedence}}}
\end{notice}

\begin{notice}{warning}{Warning:}
You must be authenticated and have perms in repository
\end{notice}

\end{fulllineitems}

\index{is\_collaborator() (pygithub3.services.repos.Collaborators method)}

\begin{fulllineitems}
\phantomsection\label{repos:pygithub3.services.repos.Collaborators.is_collaborator}\pysiglinewithargsret{\bfcode{is\_collaborator}}{\emph{collaborator}, \emph{user=None}, \emph{repo=None}}{}
Check if a user is collaborator on repository
\begin{quote}\begin{description}
\item[{Parameters}] \leavevmode\begin{itemize}
\item {} 
\textbf{collaborator} (\emph{str}) -- Collaborator's username

\item {} 
\textbf{user} (\emph{str}) -- Username

\item {} 
\textbf{repo} (\emph{str}) -- Repository

\end{itemize}

\end{description}\end{quote}

\begin{notice}{note}{Note:}
Remember {\hyperref[repos:config-precedence]{\emph{Config precedence}}}
\end{notice}

\end{fulllineitems}

\index{list() (pygithub3.services.repos.Collaborators method)}

\begin{fulllineitems}
\phantomsection\label{repos:pygithub3.services.repos.Collaborators.list}\pysiglinewithargsret{\bfcode{list}}{\emph{user=None}, \emph{repo=None}}{}
Get repository's collaborators
\begin{quote}\begin{description}
\item[{Parameters}] \leavevmode\begin{itemize}
\item {} 
\textbf{user} (\emph{str}) -- Username

\item {} 
\textbf{repo} (\emph{str}) -- Repository

\end{itemize}

\item[{Returns}] \leavevmode
A {\hyperref[result::doc]{\emph{Result}}}

\end{description}\end{quote}

\begin{notice}{note}{Note:}
Remember {\hyperref[repos:config-precedence]{\emph{Config precedence}}}
\end{notice}

\end{fulllineitems}


\end{fulllineitems}



\paragraph{Commits}
\label{repos:commits}\label{repos:commits-service}\index{Commits (class in pygithub3.services.repos)}

\begin{fulllineitems}
\phantomsection\label{repos:pygithub3.services.repos.Commits}\pysiglinewithargsret{\strong{class }\code{pygithub3.services.repos.}\bfcode{Commits}}{\emph{**config}}{}
Consume \href{http://developer.github.com/v3/repos/commits}{Commits API}

\begin{notice}{note}{Note:}
This service support {\hyperref[services:mimetypes-section]{\emph{MimeTypes}}} configuration
\end{notice}
\index{compare() (pygithub3.services.repos.Commits method)}

\begin{fulllineitems}
\phantomsection\label{repos:pygithub3.services.repos.Commits.compare}\pysiglinewithargsret{\bfcode{compare}}{\emph{base}, \emph{head}, \emph{user=None}, \emph{repo=None}}{}
Compare two commits
\begin{quote}\begin{description}
\item[{Parameters}] \leavevmode\begin{itemize}
\item {} 
\textbf{base} (\emph{str}) -- Base commit sha

\item {} 
\textbf{head} (\emph{str}) -- Head commit sha

\item {} 
\textbf{user} (\emph{str}) -- Username

\item {} 
\textbf{repo} (\emph{str}) -- Repository

\end{itemize}

\end{description}\end{quote}

\begin{notice}{note}{Note:}
Remember {\hyperref[repos:config-precedence]{\emph{Config precedence}}}
\end{notice}

\begin{Verbatim}[commandchars=\\\{\}]
\PYG{n}{commits\PYGZus{}service}\PYG{o}{.}\PYG{n}{compare}\PYG{p}{(}\PYG{l+s}{'}\PYG{l+s}{6dcb09}\PYG{l+s}{'}\PYG{p}{,} \PYG{l+s}{'}\PYG{l+s}{master}\PYG{l+s}{'}\PYG{p}{,} \PYG{n}{user}\PYG{o}{=}\PYG{l+s}{'}\PYG{l+s}{octocat}\PYG{l+s}{'}\PYG{p}{,}
    \PYG{n}{repo}\PYG{o}{=}\PYG{l+s}{'}\PYG{l+s}{oct\PYGZus{}repo}\PYG{l+s}{'}\PYG{p}{)}
\end{Verbatim}

\end{fulllineitems}

\index{create\_comment() (pygithub3.services.repos.Commits method)}

\begin{fulllineitems}
\phantomsection\label{repos:pygithub3.services.repos.Commits.create_comment}\pysiglinewithargsret{\bfcode{create\_comment}}{\emph{data}, \emph{sha}, \emph{user=None}, \emph{repo=None}}{}
Create a commit comment
\begin{quote}\begin{description}
\item[{Parameters}] \leavevmode\begin{itemize}
\item {} 
\textbf{data} (\emph{dict}) -- Input. See \href{http://developer.github.com/v3/repos/commits}{github commits doc}

\item {} 
\textbf{sha} (\emph{str}) -- Commit's sha

\item {} 
\textbf{user} (\emph{str}) -- Username

\item {} 
\textbf{repo} (\emph{str}) -- Repository

\end{itemize}

\end{description}\end{quote}

\begin{notice}{note}{Note:}
Remember {\hyperref[repos:config-precedence]{\emph{Config precedence}}}
\end{notice}

\begin{notice}{warning}{Warning:}
You must be authenticated
\end{notice}

\begin{Verbatim}[commandchars=\\\{\}]
\PYG{n}{data} \PYG{o}{=} \PYG{p}{\PYGZob{}}
    \PYG{l+s}{"}\PYG{l+s}{body}\PYG{l+s}{"}\PYG{p}{:} \PYG{l+s}{"}\PYG{l+s}{Nice change}\PYG{l+s}{"}\PYG{p}{,}
    \PYG{l+s}{"}\PYG{l+s}{commit\PYGZus{}id}\PYG{l+s}{"}\PYG{p}{:} \PYG{l+s}{"}\PYG{l+s}{6dcb09b5b57875f334f61aebed695e2e4193db5e}\PYG{l+s}{"}\PYG{p}{,}
    \PYG{l+s}{"}\PYG{l+s}{line}\PYG{l+s}{"}\PYG{p}{:} \PYG{l+m+mi}{1}\PYG{p}{,}
    \PYG{l+s}{"}\PYG{l+s}{path}\PYG{l+s}{"}\PYG{p}{:} \PYG{l+s}{"}\PYG{l+s}{file1.txt}\PYG{l+s}{"}\PYG{p}{,}
    \PYG{l+s}{"}\PYG{l+s}{position}\PYG{l+s}{"}\PYG{p}{:} \PYG{l+m+mi}{4}
\PYG{p}{\PYGZcb{}}
\PYG{n}{commits\PYGZus{}service}\PYG{o}{.}\PYG{n}{create\PYGZus{}comment}\PYG{p}{(}\PYG{n}{data}\PYG{p}{,} \PYG{l+s}{'}\PYG{l+s}{6dcb09}\PYG{l+s}{'}\PYG{p}{,} \PYG{n}{user}\PYG{o}{=}\PYG{l+s}{'}\PYG{l+s}{octocat}\PYG{l+s}{'}\PYG{p}{,}
    \PYG{n}{repo}\PYG{o}{=}\PYG{l+s}{'}\PYG{l+s}{oct\PYGZus{}repo}\PYG{l+s}{'}\PYG{p}{)}
\end{Verbatim}

\end{fulllineitems}

\index{delete\_comment() (pygithub3.services.repos.Commits method)}

\begin{fulllineitems}
\phantomsection\label{repos:pygithub3.services.repos.Commits.delete_comment}\pysiglinewithargsret{\bfcode{delete\_comment}}{\emph{cid}, \emph{user=None}, \emph{repo=None}}{}
Delete a single commit comment
\begin{quote}\begin{description}
\item[{Parameters}] \leavevmode\begin{itemize}
\item {} 
\textbf{cid} (\emph{int}) -- Commit comment id

\item {} 
\textbf{user} (\emph{str}) -- Username

\item {} 
\textbf{repo} (\emph{str}) -- Repository

\end{itemize}

\end{description}\end{quote}

\begin{notice}{note}{Note:}
Remember {\hyperref[repos:config-precedence]{\emph{Config precedence}}}
\end{notice}

\end{fulllineitems}

\index{get() (pygithub3.services.repos.Commits method)}

\begin{fulllineitems}
\phantomsection\label{repos:pygithub3.services.repos.Commits.get}\pysiglinewithargsret{\bfcode{get}}{\emph{sha}, \emph{user=None}, \emph{repo=None}}{}
Get a single commit
\begin{quote}\begin{description}
\item[{Parameters}] \leavevmode\begin{itemize}
\item {} 
\textbf{sha} (\emph{str}) -- Commit's sha

\item {} 
\textbf{user} (\emph{str}) -- Username

\item {} 
\textbf{repo} (\emph{str}) -- Repository

\end{itemize}

\end{description}\end{quote}

\begin{notice}{note}{Note:}
Remember {\hyperref[repos:config-precedence]{\emph{Config precedence}}}
\end{notice}

\end{fulllineitems}

\index{get\_comment() (pygithub3.services.repos.Commits method)}

\begin{fulllineitems}
\phantomsection\label{repos:pygithub3.services.repos.Commits.get_comment}\pysiglinewithargsret{\bfcode{get\_comment}}{\emph{cid}, \emph{user=None}, \emph{repo=None}}{}
Get a single commit comment
\begin{quote}\begin{description}
\item[{Parameters}] \leavevmode\begin{itemize}
\item {} 
\textbf{cid} (\emph{int}) -- Commit comment id

\item {} 
\textbf{user} (\emph{str}) -- Username

\item {} 
\textbf{repo} (\emph{str}) -- Repository

\end{itemize}

\end{description}\end{quote}

\begin{notice}{note}{Note:}
Remember {\hyperref[repos:config-precedence]{\emph{Config precedence}}}
\end{notice}

\end{fulllineitems}

\index{list() (pygithub3.services.repos.Commits method)}

\begin{fulllineitems}
\phantomsection\label{repos:pygithub3.services.repos.Commits.list}\pysiglinewithargsret{\bfcode{list}}{\emph{user=None}, \emph{repo=None}, \emph{sha=None}, \emph{path=None}}{}
Get repository's commits
\begin{quote}\begin{description}
\item[{Parameters}] \leavevmode\begin{itemize}
\item {} 
\textbf{user} (\emph{str}) -- Username

\item {} 
\textbf{repo} (\emph{str}) -- Repository

\item {} 
\textbf{sha} (\emph{str}) -- Sha or branch to start listing commits from

\item {} 
\textbf{path} (\emph{str}) -- Only commits containing this file path will be returned

\end{itemize}

\item[{Returns}] \leavevmode
A {\hyperref[result::doc]{\emph{Result}}}

\end{description}\end{quote}

\begin{notice}{note}{Note:}
Remember {\hyperref[repos:config-precedence]{\emph{Config precedence}}}
\end{notice}

\begin{notice}{warning}{Warning:}
Usually a repository has thousand of commits, so be careful when
consume the result. You should filter with \code{sha} or directly
clone the repository
\end{notice}

\begin{Verbatim}[commandchars=\\\{\}]
\PYG{n}{commits\PYGZus{}service}\PYG{o}{.}\PYG{n}{list}\PYG{p}{(}\PYG{n}{user}\PYG{o}{=}\PYG{l+s}{'}\PYG{l+s}{octocat}\PYG{l+s}{'}\PYG{p}{,} \PYG{n}{repo}\PYG{o}{=}\PYG{l+s}{'}\PYG{l+s}{oct\PYGZus{}repo}\PYG{l+s}{'}\PYG{p}{)}
\PYG{n}{commits\PYGZus{}service}\PYG{o}{.}\PYG{n}{list}\PYG{p}{(}\PYG{n}{user}\PYG{o}{=}\PYG{l+s}{'}\PYG{l+s}{octocat}\PYG{l+s}{'}\PYG{p}{,} \PYG{n}{repo}\PYG{o}{=}\PYG{l+s}{'}\PYG{l+s}{oct\PYGZus{}repo}\PYG{l+s}{'}\PYG{p}{,} \PYG{n}{sha}\PYG{o}{=}\PYG{l+s}{'}\PYG{l+s}{dev}\PYG{l+s}{'}\PYG{p}{)}
\PYG{n}{commits\PYGZus{}service}\PYG{o}{.}\PYG{n}{list}\PYG{p}{(}\PYG{n}{user}\PYG{o}{=}\PYG{l+s}{'}\PYG{l+s}{django}\PYG{l+s}{'}\PYG{p}{,} \PYG{n}{repo}\PYG{o}{=}\PYG{l+s}{'}\PYG{l+s}{django}\PYG{l+s}{'}\PYG{p}{,} \PYG{n}{sha}\PYG{o}{=}\PYG{l+s}{'}\PYG{l+s}{master}\PYG{l+s}{'}\PYG{p}{,}
    \PYG{n}{path}\PYG{o}{=}\PYG{l+s}{'}\PYG{l+s}{django/db/utils.py}\PYG{l+s}{'}\PYG{p}{)}
\end{Verbatim}

\end{fulllineitems}

\index{list\_comments() (pygithub3.services.repos.Commits method)}

\begin{fulllineitems}
\phantomsection\label{repos:pygithub3.services.repos.Commits.list_comments}\pysiglinewithargsret{\bfcode{list\_comments}}{\emph{sha=None}, \emph{user=None}, \emph{repo=None}}{}
Get commit's comments
\begin{quote}\begin{description}
\item[{Parameters}] \leavevmode\begin{itemize}
\item {} 
\textbf{sha} (\emph{str}) -- Commit's sha

\item {} 
\textbf{user} (\emph{str}) -- Username

\item {} 
\textbf{repo} (\emph{str}) -- Repository

\end{itemize}

\item[{Returns}] \leavevmode
A {\hyperref[result::doc]{\emph{Result}}}

\end{description}\end{quote}

\begin{notice}{note}{Note:}
Remember {\hyperref[repos:config-precedence]{\emph{Config precedence}}}
\end{notice}

If you call it without \code{sha}, get all commit's comments of a
repository

\begin{Verbatim}[commandchars=\\\{\}]
\PYG{n}{commits\PYGZus{}service}\PYG{o}{.}\PYG{n}{list\PYGZus{}comments}\PYG{p}{(}\PYG{l+s}{'}\PYG{l+s}{6dcb09}\PYG{l+s}{'}\PYG{p}{,} \PYG{n}{user}\PYG{o}{=}\PYG{l+s}{'}\PYG{l+s}{octocat}\PYG{l+s}{'}\PYG{p}{,}
    \PYG{n}{repo}\PYG{o}{=}\PYG{l+s}{'}\PYG{l+s}{oct\PYGZus{}repo}\PYG{l+s}{'}\PYG{p}{)}
\PYG{n}{commits\PYGZus{}service}\PYG{o}{.}\PYG{n}{list\PYGZus{}comments}\PYG{p}{(}\PYG{n}{user}\PYG{o}{=}\PYG{l+s}{'}\PYG{l+s}{octocat}\PYG{l+s}{'}\PYG{p}{,} \PYG{n}{repo}\PYG{o}{=}\PYG{l+s}{'}\PYG{l+s}{oct\PYGZus{}repo}\PYG{l+s}{'}\PYG{p}{)}
\end{Verbatim}

\end{fulllineitems}

\index{update\_comment() (pygithub3.services.repos.Commits method)}

\begin{fulllineitems}
\phantomsection\label{repos:pygithub3.services.repos.Commits.update_comment}\pysiglinewithargsret{\bfcode{update\_comment}}{\emph{data}, \emph{cid}, \emph{user=None}, \emph{repo=None}}{}
Update a single commit comment
\begin{quote}\begin{description}
\item[{Parameters}] \leavevmode\begin{itemize}
\item {} 
\textbf{data} (\emph{dict}) -- Input. See \href{http://developer.github.com/v3/repos/commits}{github commits doc}

\item {} 
\textbf{cid} (\emph{int}) -- Commit comment id

\item {} 
\textbf{user} (\emph{str}) -- Username

\item {} 
\textbf{repo} (\emph{str}) -- Repository

\end{itemize}

\end{description}\end{quote}

\begin{notice}{note}{Note:}
Remember {\hyperref[repos:config-precedence]{\emph{Config precedence}}}
\end{notice}

\begin{notice}{warning}{Warning:}
You must be authenticated
\end{notice}

\begin{Verbatim}[commandchars=\\\{\}]
\PYG{n}{commits\PYGZus{}service}\PYG{o}{.}\PYG{n}{update\PYGZus{}comment}\PYG{p}{(}\PYG{n+nb}{dict}\PYG{p}{(}\PYG{n}{body}\PYG{o}{=}\PYG{l+s}{'}\PYG{l+s}{nice change}\PYG{l+s}{'}\PYG{p}{)}\PYG{p}{,} \PYG{l+m+mi}{42}\PYG{p}{,}
    \PYG{n}{user}\PYG{o}{=}\PYG{l+s}{'}\PYG{l+s}{octocat}\PYG{l+s}{'}\PYG{p}{,} \PYG{n}{repo}\PYG{o}{=}\PYG{l+s}{'}\PYG{l+s}{oct\PYGZus{}repo}\PYG{l+s}{'}\PYG{p}{)}
\end{Verbatim}

\end{fulllineitems}


\end{fulllineitems}



\paragraph{Downloads}
\label{repos:downloads}\label{repos:downloads-service}\index{Downloads (class in pygithub3.services.repos)}

\begin{fulllineitems}
\phantomsection\label{repos:pygithub3.services.repos.Downloads}\pysiglinewithargsret{\strong{class }\code{pygithub3.services.repos.}\bfcode{Downloads}}{\emph{**config}}{}
Consume \href{http://developer.github.com/v3/repos/downloads}{Downloads API}
\index{create() (pygithub3.services.repos.Downloads method)}

\begin{fulllineitems}
\phantomsection\label{repos:pygithub3.services.repos.Downloads.create}\pysiglinewithargsret{\bfcode{create}}{\emph{data}, \emph{user=None}, \emph{repo=None}}{}
Create a new download
\begin{quote}\begin{description}
\item[{Parameters}] \leavevmode\begin{itemize}
\item {} 
\textbf{data} (\emph{dict}) -- Input. See \href{http://developer.github.com/v3/repos/downloads}{github downloads doc}

\item {} 
\textbf{user} (\emph{str}) -- Username

\item {} 
\textbf{repo} (\emph{str}) -- Repository

\end{itemize}

\end{description}\end{quote}

\begin{notice}{note}{Note:}
Remember {\hyperref[repos:config-precedence]{\emph{Config precedence}}}
\end{notice}

It is a two step process. After you create the download, you must
call the \code{upload} function of \code{Download} resource with
\code{file\_path}

\begin{notice}{warning}{Warning:}
In \emph{alpha} state
\end{notice}

\begin{Verbatim}[commandchars=\\\{\}]
\PYG{c}{\PYGZsh{} Step 1}
\PYG{n}{download} \PYG{o}{=} \PYG{n}{downloads\PYGZus{}service}\PYG{o}{.}\PYG{n}{create}\PYG{p}{(}
    \PYG{n+nb}{dict}\PYG{p}{(}\PYG{n}{name}\PYG{o}{=}\PYG{l+s}{'}\PYG{l+s}{new\PYGZus{}download}\PYG{l+s}{'}\PYG{p}{,} \PYG{n}{size}\PYG{o}{=}\PYG{l+m+mi}{1130}\PYG{p}{)}\PYG{p}{,}
    \PYG{n}{user}\PYG{o}{=}\PYG{l+s}{'}\PYG{l+s}{octocat}\PYG{l+s}{'}\PYG{p}{,} \PYG{n}{repo}\PYG{o}{=}\PYG{l+s}{'}\PYG{l+s}{oct\PYGZus{}repo}\PYG{l+s}{'}\PYG{p}{)}

\PYG{c}{\PYGZsh{} Step 2}
\PYG{n}{download}\PYG{o}{.}\PYG{n}{upload}\PYG{p}{(}\PYG{l+s}{'}\PYG{l+s}{/home/user/file.ext}\PYG{l+s}{'}\PYG{p}{)}
\end{Verbatim}

\end{fulllineitems}

\index{delete() (pygithub3.services.repos.Downloads method)}

\begin{fulllineitems}
\phantomsection\label{repos:pygithub3.services.repos.Downloads.delete}\pysiglinewithargsret{\bfcode{delete}}{\emph{id}, \emph{user=None}, \emph{repo=None}}{}
Delete a download
\begin{quote}\begin{description}
\item[{Parameters}] \leavevmode\begin{itemize}
\item {} 
\textbf{id} (\emph{int}) -- Download id

\item {} 
\textbf{user} (\emph{str}) -- Username

\item {} 
\textbf{repo} (\emph{str}) -- Repository

\end{itemize}

\end{description}\end{quote}

\begin{notice}{note}{Note:}
Remember {\hyperref[repos:config-precedence]{\emph{Config precedence}}}
\end{notice}

\end{fulllineitems}

\index{get() (pygithub3.services.repos.Downloads method)}

\begin{fulllineitems}
\phantomsection\label{repos:pygithub3.services.repos.Downloads.get}\pysiglinewithargsret{\bfcode{get}}{\emph{id}, \emph{user=None}, \emph{repo=None}}{}
Get a single download
\begin{quote}\begin{description}
\item[{Parameters}] \leavevmode\begin{itemize}
\item {} 
\textbf{id} (\emph{int}) -- Download id

\item {} 
\textbf{user} (\emph{str}) -- Username

\item {} 
\textbf{repo} (\emph{str}) -- Repository

\end{itemize}

\end{description}\end{quote}

\begin{notice}{note}{Note:}
Remember {\hyperref[repos:config-precedence]{\emph{Config precedence}}}
\end{notice}

\end{fulllineitems}

\index{list() (pygithub3.services.repos.Downloads method)}

\begin{fulllineitems}
\phantomsection\label{repos:pygithub3.services.repos.Downloads.list}\pysiglinewithargsret{\bfcode{list}}{\emph{user=None}, \emph{repo=None}}{}
Get repository's downloads
\begin{quote}\begin{description}
\item[{Parameters}] \leavevmode\begin{itemize}
\item {} 
\textbf{user} (\emph{str}) -- Username

\item {} 
\textbf{repo} (\emph{str}) -- Repository

\end{itemize}

\item[{Returns}] \leavevmode
A {\hyperref[result::doc]{\emph{Result}}}

\end{description}\end{quote}

\begin{notice}{note}{Note:}
Remember {\hyperref[repos:config-precedence]{\emph{Config precedence}}}
\end{notice}

\end{fulllineitems}


\end{fulllineitems}



\paragraph{Forks}
\label{repos:forks}\label{repos:forks-service}\index{Forks (class in pygithub3.services.repos)}

\begin{fulllineitems}
\phantomsection\label{repos:pygithub3.services.repos.Forks}\pysiglinewithargsret{\strong{class }\code{pygithub3.services.repos.}\bfcode{Forks}}{\emph{**config}}{}
Consume \href{http://developer.github.com/v3/repos/forks}{Forks API}
\index{create() (pygithub3.services.repos.Forks method)}

\begin{fulllineitems}
\phantomsection\label{repos:pygithub3.services.repos.Forks.create}\pysiglinewithargsret{\bfcode{create}}{\emph{user=None}, \emph{repo=None}, \emph{org=None}}{}
Fork a repository
\begin{quote}\begin{description}
\item[{Parameters}] \leavevmode\begin{itemize}
\item {} 
\textbf{user} (\emph{str}) -- Username

\item {} 
\textbf{repo} (\emph{str}) -- Repository

\item {} 
\textbf{org} (\emph{str}) -- Organization name (optional)

\end{itemize}

\end{description}\end{quote}

\begin{notice}{note}{Note:}
Remember {\hyperref[repos:config-precedence]{\emph{Config precedence}}}
\end{notice}

\begin{notice}{warning}{Warning:}
You must be authenticated
\end{notice}

If you call it with \code{org}, the repository will be forked into this
organization.

\begin{Verbatim}[commandchars=\\\{\}]
\PYG{n}{forks\PYGZus{}service}\PYG{o}{.}\PYG{n}{create}\PYG{p}{(}\PYG{n}{user}\PYG{o}{=}\PYG{l+s}{'}\PYG{l+s}{octocat}\PYG{l+s}{'}\PYG{p}{,} \PYG{n}{repo}\PYG{o}{=}\PYG{l+s}{'}\PYG{l+s}{oct\PYGZus{}repo}\PYG{l+s}{'}\PYG{p}{)}
\PYG{n}{forks\PYGZus{}service}\PYG{o}{.}\PYG{n}{create}\PYG{p}{(}\PYG{n}{user}\PYG{o}{=}\PYG{l+s}{'}\PYG{l+s}{octocat}\PYG{l+s}{'}\PYG{p}{,} \PYG{n}{repo}\PYG{o}{=}\PYG{l+s}{'}\PYG{l+s}{oct\PYGZus{}repo}\PYG{l+s}{'}\PYG{p}{,}
    \PYG{n}{org}\PYG{o}{=}\PYG{l+s}{'}\PYG{l+s}{myorganization}\PYG{l+s}{'}\PYG{p}{)}
\end{Verbatim}

\end{fulllineitems}

\index{list() (pygithub3.services.repos.Forks method)}

\begin{fulllineitems}
\phantomsection\label{repos:pygithub3.services.repos.Forks.list}\pysiglinewithargsret{\bfcode{list}}{\emph{user=None}, \emph{repo=None}, \emph{sort='newest'}}{}
Get repository's forks
\begin{quote}\begin{description}
\item[{Parameters}] \leavevmode\begin{itemize}
\item {} 
\textbf{user} (\emph{str}) -- Username

\item {} 
\textbf{repo} (\emph{str}) -- Repository

\item {} 
\textbf{sort} (\emph{str}) -- Order resources (optional). See \href{http://developer.github.com/v3/repos/forks}{github forks doc}

\end{itemize}

\item[{Returns}] \leavevmode
A {\hyperref[result::doc]{\emph{Result}}}

\end{description}\end{quote}

\begin{notice}{note}{Note:}
Remember {\hyperref[repos:config-precedence]{\emph{Config precedence}}}
\end{notice}

\begin{Verbatim}[commandchars=\\\{\}]
\PYG{n}{forks\PYGZus{}service}\PYG{o}{.}\PYG{n}{list}\PYG{p}{(}\PYG{n}{user}\PYG{o}{=}\PYG{l+s}{'}\PYG{l+s}{octocat}\PYG{l+s}{'}\PYG{p}{,} \PYG{n}{repo}\PYG{o}{=}\PYG{l+s}{'}\PYG{l+s}{oct\PYGZus{}repo}\PYG{l+s}{'}\PYG{p}{,} \PYG{n}{sort}\PYG{o}{=}\PYG{l+s}{'}\PYG{l+s}{oldest}\PYG{l+s}{'}\PYG{p}{)}
\end{Verbatim}

\end{fulllineitems}


\end{fulllineitems}



\paragraph{Keys}
\label{repos:keys}\label{repos:repokeys-service}\index{Keys (class in pygithub3.services.repos)}

\begin{fulllineitems}
\phantomsection\label{repos:pygithub3.services.repos.Keys}\pysiglinewithargsret{\strong{class }\code{pygithub3.services.repos.}\bfcode{Keys}}{\emph{**config}}{}
Consume \href{http://developer.github.com/v3/repos/keys}{Deploy keys API}
\index{create() (pygithub3.services.repos.Keys method)}

\begin{fulllineitems}
\phantomsection\label{repos:pygithub3.services.repos.Keys.create}\pysiglinewithargsret{\bfcode{create}}{\emph{data}, \emph{user=None}, \emph{repo=None}}{}
Create a repository key
\begin{quote}\begin{description}
\item[{Parameters}] \leavevmode\begin{itemize}
\item {} 
\textbf{data} (\emph{dict}) -- Input. See \href{http://developer.github.com/v3/repos/keys}{github keys doc}

\item {} 
\textbf{user} (\emph{str}) -- Username

\item {} 
\textbf{repo} (\emph{str}) -- Repository

\end{itemize}

\end{description}\end{quote}

\begin{notice}{note}{Note:}
Remember {\hyperref[repos:config-precedence]{\emph{Config precedence}}}
\end{notice}

\begin{notice}{warning}{Warning:}
You must be authenticated and have perms in the repository
\end{notice}

\begin{Verbatim}[commandchars=\\\{\}]
\PYG{n}{keys\PYGZus{}service}\PYG{o}{.}\PYG{n}{create}\PYG{p}{(}\PYG{n+nb}{dict}\PYG{p}{(}\PYG{n}{title}\PYG{o}{=}\PYG{l+s}{'}\PYG{l+s}{new key}\PYG{l+s}{'}\PYG{p}{,} \PYG{n}{key}\PYG{o}{=}\PYG{l+s}{'}\PYG{l+s}{ssh-rsa AAA...}\PYG{l+s}{'}\PYG{p}{)}\PYG{p}{,}
    \PYG{n}{user}\PYG{o}{=}\PYG{l+s}{'}\PYG{l+s}{octocat}\PYG{l+s}{'}\PYG{p}{,} \PYG{n}{repo}\PYG{o}{=}\PYG{l+s}{'}\PYG{l+s}{oct\PYGZus{}repo}\PYG{l+s}{'}\PYG{p}{)}
\end{Verbatim}

\end{fulllineitems}

\index{delete() (pygithub3.services.repos.Keys method)}

\begin{fulllineitems}
\phantomsection\label{repos:pygithub3.services.repos.Keys.delete}\pysiglinewithargsret{\bfcode{delete}}{\emph{id}, \emph{user=None}, \emph{repo=None}}{}
Delete a repository key
\begin{quote}\begin{description}
\item[{Parameters}] \leavevmode\begin{itemize}
\item {} 
\textbf{id} (\emph{int}) -- Repository key id

\item {} 
\textbf{user} (\emph{str}) -- Username

\item {} 
\textbf{repo} (\emph{str}) -- Repository

\end{itemize}

\end{description}\end{quote}

\begin{notice}{note}{Note:}
Remember {\hyperref[repos:config-precedence]{\emph{Config precedence}}}
\end{notice}

\end{fulllineitems}

\index{get() (pygithub3.services.repos.Keys method)}

\begin{fulllineitems}
\phantomsection\label{repos:pygithub3.services.repos.Keys.get}\pysiglinewithargsret{\bfcode{get}}{\emph{id}, \emph{user=None}, \emph{repo=None}}{}
Get a single repository key
\begin{quote}\begin{description}
\item[{Parameters}] \leavevmode\begin{itemize}
\item {} 
\textbf{id} (\emph{int}) -- Repository key id

\item {} 
\textbf{user} (\emph{str}) -- Username

\item {} 
\textbf{repo} (\emph{str}) -- Repository

\end{itemize}

\end{description}\end{quote}

\begin{notice}{note}{Note:}
Remember {\hyperref[repos:config-precedence]{\emph{Config precedence}}}
\end{notice}

\end{fulllineitems}

\index{list() (pygithub3.services.repos.Keys method)}

\begin{fulllineitems}
\phantomsection\label{repos:pygithub3.services.repos.Keys.list}\pysiglinewithargsret{\bfcode{list}}{\emph{user=None}, \emph{repo=None}}{}
Get repository's keys
\begin{quote}\begin{description}
\item[{Parameters}] \leavevmode\begin{itemize}
\item {} 
\textbf{user} (\emph{str}) -- Username

\item {} 
\textbf{repo} (\emph{str}) -- Repository

\end{itemize}

\item[{Returns}] \leavevmode
A {\hyperref[result::doc]{\emph{Result}}}

\end{description}\end{quote}

\begin{notice}{note}{Note:}
Remember {\hyperref[repos:config-precedence]{\emph{Config precedence}}}
\end{notice}

\end{fulllineitems}

\index{update() (pygithub3.services.repos.Keys method)}

\begin{fulllineitems}
\phantomsection\label{repos:pygithub3.services.repos.Keys.update}\pysiglinewithargsret{\bfcode{update}}{\emph{id}, \emph{data}, \emph{user=None}, \emph{repo=None}}{}
Update a repository key
\begin{quote}\begin{description}
\item[{Parameters}] \leavevmode\begin{itemize}
\item {} 
\textbf{id} (\emph{int}) -- Repository key id

\item {} 
\textbf{data} (\emph{dict}) -- Input. See \href{http://developer.github.com/v3/repos/keys}{github keys doc}

\item {} 
\textbf{user} (\emph{str}) -- Username

\item {} 
\textbf{repo} (\emph{str}) -- Repository

\end{itemize}

\end{description}\end{quote}

\begin{notice}{note}{Note:}
Remember {\hyperref[repos:config-precedence]{\emph{Config precedence}}}
\end{notice}

\begin{notice}{warning}{Warning:}
You must be authenticated and have perms in the repository
\end{notice}

\begin{Verbatim}[commandchars=\\\{\}]
\PYG{n}{keys\PYGZus{}service}\PYG{o}{.}\PYG{n}{update}\PYG{p}{(}\PYG{l+m+mi}{42}\PYG{p}{,} \PYG{n+nb}{dict}\PYG{p}{(}\PYG{n}{title}\PYG{o}{=}\PYG{l+s}{'}\PYG{l+s}{new title}\PYG{l+s}{'}\PYG{p}{)}\PYG{p}{,}
    \PYG{n}{user}\PYG{o}{=}\PYG{l+s}{'}\PYG{l+s}{octocat}\PYG{l+s}{'}\PYG{p}{,} \PYG{n}{repo}\PYG{o}{=}\PYG{l+s}{'}\PYG{l+s}{oct\PYGZus{}repo}\PYG{l+s}{'}\PYG{p}{)}
\end{Verbatim}

\end{fulllineitems}


\end{fulllineitems}



\paragraph{Watchers}
\label{repos:watchers}\label{repos:watchers-service}\index{Watchers (class in pygithub3.services.repos)}

\begin{fulllineitems}
\phantomsection\label{repos:pygithub3.services.repos.Watchers}\pysiglinewithargsret{\strong{class }\code{pygithub3.services.repos.}\bfcode{Watchers}}{\emph{**config}}{}
Consume \href{http://developer.github.com/v3/repos/watching}{Watching API}
\index{is\_watching() (pygithub3.services.repos.Watchers method)}

\begin{fulllineitems}
\phantomsection\label{repos:pygithub3.services.repos.Watchers.is_watching}\pysiglinewithargsret{\bfcode{is\_watching}}{\emph{user=None}, \emph{repo=None}}{}
Check if authenticated user is watching a repository
\begin{quote}\begin{description}
\item[{Parameters}] \leavevmode\begin{itemize}
\item {} 
\textbf{user} (\emph{str}) -- Username

\item {} 
\textbf{repo} (\emph{str}) -- Repository

\end{itemize}

\end{description}\end{quote}

\begin{notice}{note}{Note:}
Remember {\hyperref[repos:config-precedence]{\emph{Config precedence}}}
\end{notice}

\begin{notice}{warning}{Warning:}
You must be authenticated
\end{notice}

\end{fulllineitems}

\index{list() (pygithub3.services.repos.Watchers method)}

\begin{fulllineitems}
\phantomsection\label{repos:pygithub3.services.repos.Watchers.list}\pysiglinewithargsret{\bfcode{list}}{\emph{user=None}, \emph{repo=None}}{}
Get repository's watchers
\begin{quote}\begin{description}
\item[{Parameters}] \leavevmode\begin{itemize}
\item {} 
\textbf{user} (\emph{str}) -- Username

\item {} 
\textbf{repo} (\emph{str}) -- Repository

\end{itemize}

\item[{Returns}] \leavevmode
A {\hyperref[result::doc]{\emph{Result}}}

\end{description}\end{quote}

\begin{notice}{note}{Note:}
Remember {\hyperref[repos:config-precedence]{\emph{Config precedence}}}
\end{notice}

\end{fulllineitems}

\index{list\_repos() (pygithub3.services.repos.Watchers method)}

\begin{fulllineitems}
\phantomsection\label{repos:pygithub3.services.repos.Watchers.list_repos}\pysiglinewithargsret{\bfcode{list\_repos}}{\emph{user=None}}{}
Get repositories being watched by a user
\begin{quote}\begin{description}
\item[{Parameters}] \leavevmode
\textbf{user} (\emph{str}) -- Username

\item[{Returns}] \leavevmode
A {\hyperref[result::doc]{\emph{Result}}}

\end{description}\end{quote}

If you call it without user and you are authenticated, get the
repositories being watched by the authenticated user.

\begin{notice}{warning}{Warning:}
If you aren't authenticated and call without user, it returns 403
\end{notice}

\begin{Verbatim}[commandchars=\\\{\}]
\PYG{n}{watchers\PYGZus{}service}\PYG{o}{.}\PYG{n}{list\PYGZus{}repos}\PYG{p}{(}\PYG{l+s}{'}\PYG{l+s}{copitux}\PYG{l+s}{'}\PYG{p}{)}
\PYG{n}{watchers\PYGZus{}service}\PYG{o}{.}\PYG{n}{list\PYGZus{}repos}\PYG{p}{(}\PYG{p}{)}
\end{Verbatim}

\end{fulllineitems}

\index{unwatch() (pygithub3.services.repos.Watchers method)}

\begin{fulllineitems}
\phantomsection\label{repos:pygithub3.services.repos.Watchers.unwatch}\pysiglinewithargsret{\bfcode{unwatch}}{\emph{user=None}, \emph{repo=None}}{}
Stop watching a repository
\begin{quote}\begin{description}
\item[{Parameters}] \leavevmode\begin{itemize}
\item {} 
\textbf{user} (\emph{str}) -- Username

\item {} 
\textbf{repo} (\emph{str}) -- Repository

\end{itemize}

\end{description}\end{quote}

\begin{notice}{note}{Note:}
Remember {\hyperref[repos:config-precedence]{\emph{Config precedence}}}
\end{notice}

\begin{notice}{warning}{Warning:}
You must be authenticated
\end{notice}

\end{fulllineitems}

\index{watch() (pygithub3.services.repos.Watchers method)}

\begin{fulllineitems}
\phantomsection\label{repos:pygithub3.services.repos.Watchers.watch}\pysiglinewithargsret{\bfcode{watch}}{\emph{user=None}, \emph{repo=None}}{}
Watch a repository
\begin{quote}\begin{description}
\item[{Parameters}] \leavevmode\begin{itemize}
\item {} 
\textbf{user} (\emph{str}) -- Username

\item {} 
\textbf{repo} (\emph{str}) -- Repository

\end{itemize}

\end{description}\end{quote}

\begin{notice}{note}{Note:}
Remember {\hyperref[repos:config-precedence]{\emph{Config precedence}}}
\end{notice}

\begin{notice}{warning}{Warning:}
You must be authenticated
\end{notice}

\end{fulllineitems}


\end{fulllineitems}



\paragraph{Hooks}
\label{repos:hooks}\label{repos:hooks-service}\index{Hooks (class in pygithub3.services.repos)}

\begin{fulllineitems}
\phantomsection\label{repos:pygithub3.services.repos.Hooks}\pysiglinewithargsret{\strong{class }\code{pygithub3.services.repos.}\bfcode{Hooks}}{\emph{**config}}{}
Consume \href{http://developer.github.com/v3/repos/hooks}{Hooks API}

\begin{notice}{warning}{Warning:}
You must be authenticated and have repository's admin-permission
\end{notice}
\index{create() (pygithub3.services.repos.Hooks method)}

\begin{fulllineitems}
\phantomsection\label{repos:pygithub3.services.repos.Hooks.create}\pysiglinewithargsret{\bfcode{create}}{\emph{data}, \emph{user=None}, \emph{repo=None}}{}
Create a hook
\begin{quote}\begin{description}
\item[{Parameters}] \leavevmode\begin{itemize}
\item {} 
\textbf{data} (\emph{dict}) -- Input. See \href{http://developer.github.com/v3/repos/hooks}{github hooks doc}

\item {} 
\textbf{user} (\emph{str}) -- Username

\item {} 
\textbf{repo} (\emph{str}) -- Repository

\end{itemize}

\end{description}\end{quote}

\begin{notice}{note}{Note:}
Remember {\hyperref[repos:config-precedence]{\emph{Config precedence}}}
\end{notice}

\begin{Verbatim}[commandchars=\\\{\}]
\PYG{n}{data} \PYG{o}{=} \PYG{p}{\PYGZob{}}
    \PYG{l+s}{"}\PYG{l+s}{name}\PYG{l+s}{"}\PYG{p}{:} \PYG{l+s}{"}\PYG{l+s}{acunote}\PYG{l+s}{"}\PYG{p}{,}
    \PYG{l+s}{"}\PYG{l+s}{active}\PYG{l+s}{"}\PYG{p}{:} \PYG{n+nb+bp}{True}\PYG{p}{,}
    \PYG{l+s}{"}\PYG{l+s}{config}\PYG{l+s}{"}\PYG{p}{:} \PYG{p}{\PYGZob{}}
        \PYG{l+s}{'}\PYG{l+s}{token}\PYG{l+s}{'}\PYG{p}{:} \PYG{l+s}{'}\PYG{l+s}{AAA...}\PYG{l+s}{'}\PYG{p}{,}
    \PYG{p}{\PYGZcb{}}\PYG{p}{,}
    \PYG{l+s}{"}\PYG{l+s}{events}\PYG{l+s}{"}\PYG{p}{:} \PYG{p}{[}\PYG{l+s}{'}\PYG{l+s}{push}\PYG{l+s}{'}\PYG{p}{,} \PYG{l+s}{'}\PYG{l+s}{issues}\PYG{l+s}{'}\PYG{p}{]}\PYG{p}{,}
\PYG{p}{\PYGZcb{}}
\PYG{n}{hooks\PYGZus{}service}\PYG{o}{.}\PYG{n}{create}\PYG{p}{(}\PYG{n}{data}\PYG{p}{,} \PYG{n}{user}\PYG{o}{=}\PYG{l+s}{'}\PYG{l+s}{octocat}\PYG{l+s}{'}\PYG{p}{,} \PYG{n}{repo}\PYG{o}{=}\PYG{l+s}{'}\PYG{l+s}{oct\PYGZus{}repo}\PYG{l+s}{'}\PYG{p}{)}
\end{Verbatim}

\end{fulllineitems}

\index{delete() (pygithub3.services.repos.Hooks method)}

\begin{fulllineitems}
\phantomsection\label{repos:pygithub3.services.repos.Hooks.delete}\pysiglinewithargsret{\bfcode{delete}}{\emph{hook\_id}, \emph{user=None}, \emph{repo=None}}{}
Delete a single hook
\begin{quote}\begin{description}
\item[{Parameters}] \leavevmode\begin{itemize}
\item {} 
\textbf{hook\_id} (\emph{int}) -- Hook id

\item {} 
\textbf{user} (\emph{str}) -- Username

\item {} 
\textbf{repo} (\emph{str}) -- Repository

\end{itemize}

\end{description}\end{quote}

\begin{notice}{note}{Note:}
Remember {\hyperref[repos:config-precedence]{\emph{Config precedence}}}
\end{notice}

\end{fulllineitems}

\index{get() (pygithub3.services.repos.Hooks method)}

\begin{fulllineitems}
\phantomsection\label{repos:pygithub3.services.repos.Hooks.get}\pysiglinewithargsret{\bfcode{get}}{\emph{hook\_id}, \emph{user=None}, \emph{repo=None}}{}
Get a single hook
\begin{quote}\begin{description}
\item[{Parameters}] \leavevmode\begin{itemize}
\item {} 
\textbf{hook\_id} (\emph{int}) -- Hook id

\item {} 
\textbf{user} (\emph{str}) -- Username

\item {} 
\textbf{repo} (\emph{str}) -- Repository

\end{itemize}

\end{description}\end{quote}

\begin{notice}{note}{Note:}
Remember {\hyperref[repos:config-precedence]{\emph{Config precedence}}}
\end{notice}

\end{fulllineitems}

\index{list() (pygithub3.services.repos.Hooks method)}

\begin{fulllineitems}
\phantomsection\label{repos:pygithub3.services.repos.Hooks.list}\pysiglinewithargsret{\bfcode{list}}{\emph{user=None}, \emph{repo=None}}{}
Get repository's hooks
\begin{quote}\begin{description}
\item[{Parameters}] \leavevmode\begin{itemize}
\item {} 
\textbf{user} (\emph{str}) -- Username

\item {} 
\textbf{repo} (\emph{str}) -- Repository

\end{itemize}

\item[{Returns}] \leavevmode
A {\hyperref[result::doc]{\emph{Result}}}

\end{description}\end{quote}

\begin{notice}{note}{Note:}
Remember {\hyperref[repos:config-precedence]{\emph{Config precedence}}}
\end{notice}

\end{fulllineitems}

\index{test() (pygithub3.services.repos.Hooks method)}

\begin{fulllineitems}
\phantomsection\label{repos:pygithub3.services.repos.Hooks.test}\pysiglinewithargsret{\bfcode{test}}{\emph{hook\_id}, \emph{user=None}, \emph{repo=None}}{}
Test a hook
\begin{quote}\begin{description}
\item[{Parameters}] \leavevmode\begin{itemize}
\item {} 
\textbf{user} (\emph{str}) -- Username

\item {} 
\textbf{repo} (\emph{str}) -- Repository

\end{itemize}

\end{description}\end{quote}

\begin{notice}{note}{Note:}
Remember {\hyperref[repos:config-precedence]{\emph{Config precedence}}}
\end{notice}

This will trigger the hook with the latest push to the current
repository.

\end{fulllineitems}

\index{update() (pygithub3.services.repos.Hooks method)}

\begin{fulllineitems}
\phantomsection\label{repos:pygithub3.services.repos.Hooks.update}\pysiglinewithargsret{\bfcode{update}}{\emph{hook\_id}, \emph{data}, \emph{user=None}, \emph{repo=None}}{}
Update a single hook
\begin{quote}\begin{description}
\item[{Parameters}] \leavevmode\begin{itemize}
\item {} 
\textbf{hook\_id} (\emph{int}) -- Hook id

\item {} 
\textbf{data} (\emph{dict}) -- Input. See \href{http://developer.github.com/v3/repos/hooks}{github hooks doc}

\item {} 
\textbf{user} (\emph{str}) -- Username

\item {} 
\textbf{repo} (\emph{str}) -- Repository

\end{itemize}

\end{description}\end{quote}

\begin{notice}{note}{Note:}
Remember {\hyperref[repos:config-precedence]{\emph{Config precedence}}}
\end{notice}

\begin{Verbatim}[commandchars=\\\{\}]
\PYG{n}{hooks\PYGZus{}service}\PYG{o}{.}\PYG{n}{update}\PYG{p}{(}\PYG{l+m+mi}{42}\PYG{p}{,} \PYG{n+nb}{dict}\PYG{p}{(}\PYG{n}{active}\PYG{o}{=}\PYG{n+nb+bp}{False}\PYG{p}{)}\PYG{p}{,} \PYG{n}{user}\PYG{o}{=}\PYG{l+s}{'}\PYG{l+s}{octocat}\PYG{l+s}{'}\PYG{p}{,}
    \PYG{n}{repo}\PYG{o}{=}\PYG{l+s}{'}\PYG{l+s}{oct\PYGZus{}repo}\PYG{l+s}{'}\PYG{p}{)}
\end{Verbatim}

\end{fulllineitems}


\end{fulllineitems}



\subsubsection{Gists services}
\label{gists:gists-service}\label{gists:github-hooks-doc}\label{gists::doc}\label{gists:gists-services}
\textbf{Fast sample}:

\begin{Verbatim}[commandchars=\\\{\}]
\PYG{k+kn}{from} \PYG{n+nn}{pygithub3} \PYG{k+kn}{import} \PYG{n}{Github}

\PYG{n}{auth} \PYG{o}{=} \PYG{n+nb}{dict}\PYG{p}{(}\PYG{n}{login}\PYG{o}{=}\PYG{l+s}{'}\PYG{l+s}{octocat}\PYG{l+s}{'}\PYG{p}{,} \PYG{n}{password}\PYG{o}{=}\PYG{l+s}{'}\PYG{l+s}{pass}\PYG{l+s}{'}\PYG{p}{)}
\PYG{n}{gh} \PYG{o}{=} \PYG{n}{Github}\PYG{p}{(}\PYG{o}{*}\PYG{o}{*}\PYG{n}{auth}\PYG{p}{)}

\PYG{n}{octocat\PYGZus{}gists} \PYG{o}{=} \PYG{n}{gh}\PYG{o}{.}\PYG{n}{gists}\PYG{o}{.}\PYG{n}{list}\PYG{p}{(}\PYG{p}{)}
\PYG{n}{the\PYGZus{}first\PYGZus{}gist} \PYG{o}{=} \PYG{n}{gh}\PYG{o}{.}\PYG{n}{gists}\PYG{o}{.}\PYG{n}{get}\PYG{p}{(}\PYG{l+m+mi}{1}\PYG{p}{)}

\PYG{n}{the\PYGZus{}first\PYGZus{}gist\PYGZus{}comments} \PYG{o}{=} \PYG{n}{gh}\PYG{o}{.}\PYG{n}{gists}\PYG{o}{.}\PYG{n}{comments}\PYG{o}{.}\PYG{n}{list}\PYG{p}{(}\PYG{l+m+mi}{1}\PYG{p}{)}
\end{Verbatim}


\paragraph{Gist}
\label{gists:gist}\index{Gist (class in pygithub3.services.gists)}

\begin{fulllineitems}
\phantomsection\label{gists:pygithub3.services.gists.Gist}\pysiglinewithargsret{\strong{class }\code{pygithub3.services.gists.}\bfcode{Gist}}{\emph{**config}}{}
Consume \href{http://developer.github.com/v3/gists}{Gists API}
\index{comments (Gist attribute)}

\begin{fulllineitems}
\phantomsection\label{gists:Gist.comments}\pysigline{\bfcode{comments}}
{\hyperref[gists:comments-service]{\emph{Comments}}}

\end{fulllineitems}

\index{create() (pygithub3.services.gists.Gist method)}

\begin{fulllineitems}
\phantomsection\label{gists:pygithub3.services.gists.Gist.create}\pysiglinewithargsret{\bfcode{create}}{\emph{data}}{}
Create a gist
\begin{quote}\begin{description}
\item[{Parameters}] \leavevmode
\textbf{data} (\emph{dict}) -- Input. See \href{http://developer.github.com/v3/gists}{github gists doc}

\end{description}\end{quote}

\begin{Verbatim}[commandchars=\\\{\}]
\PYG{n}{gist\PYGZus{}service}\PYG{o}{.}\PYG{n}{create}\PYG{p}{(}\PYG{n+nb}{dict}\PYG{p}{(}\PYG{n}{description}\PYG{o}{=}\PYG{l+s}{'}\PYG{l+s}{some gist}\PYG{l+s}{'}\PYG{p}{,} \PYG{n}{public}\PYG{o}{=}\PYG{n+nb+bp}{True}\PYG{p}{,}
    \PYG{n}{files}\PYG{o}{=}\PYG{p}{\PYGZob{}}\PYG{l+s}{'}\PYG{l+s}{xample.py}\PYG{l+s}{'}\PYG{p}{:} \PYG{p}{\PYGZob{}}\PYG{l+s}{'}\PYG{l+s}{content}\PYG{l+s}{'}\PYG{p}{:} \PYG{l+s}{'}\PYG{l+s}{import code}\PYG{l+s}{'}\PYG{p}{\PYGZcb{}}\PYG{p}{\PYGZcb{}}\PYG{p}{)}\PYG{p}{)}
\end{Verbatim}

\end{fulllineitems}

\index{delete() (pygithub3.services.gists.Gist method)}

\begin{fulllineitems}
\phantomsection\label{gists:pygithub3.services.gists.Gist.delete}\pysiglinewithargsret{\bfcode{delete}}{\emph{id}}{}
Delete a gist
\begin{quote}\begin{description}
\item[{Parameters}] \leavevmode
\textbf{id} (\emph{int}) -- Gist id

\end{description}\end{quote}

\begin{notice}{warning}{Warning:}
You must be authenticated
\end{notice}

\end{fulllineitems}

\index{fork() (pygithub3.services.gists.Gist method)}

\begin{fulllineitems}
\phantomsection\label{gists:pygithub3.services.gists.Gist.fork}\pysiglinewithargsret{\bfcode{fork}}{\emph{id}}{}
Fork a gist
\begin{quote}\begin{description}
\item[{Parameters}] \leavevmode
\textbf{id} (\emph{int}) -- Gist id

\end{description}\end{quote}

\begin{notice}{warning}{Warning:}
You must be authenticated
\end{notice}

\end{fulllineitems}

\index{get() (pygithub3.services.gists.Gist method)}

\begin{fulllineitems}
\phantomsection\label{gists:pygithub3.services.gists.Gist.get}\pysiglinewithargsret{\bfcode{get}}{\emph{id}}{}
Get a single gist
\begin{quote}\begin{description}
\item[{Parameters}] \leavevmode
\textbf{id} (\emph{int}) -- Gist id

\end{description}\end{quote}

\end{fulllineitems}

\index{is\_starred() (pygithub3.services.gists.Gist method)}

\begin{fulllineitems}
\phantomsection\label{gists:pygithub3.services.gists.Gist.is_starred}\pysiglinewithargsret{\bfcode{is\_starred}}{\emph{id}}{}
Check if a gist is starred
\begin{quote}\begin{description}
\item[{Parameters}] \leavevmode
\textbf{id} (\emph{int}) -- Gist id

\end{description}\end{quote}

\begin{notice}{warning}{Warning:}
You must be authenticated
\end{notice}

\end{fulllineitems}

\index{list() (pygithub3.services.gists.Gist method)}

\begin{fulllineitems}
\phantomsection\label{gists:pygithub3.services.gists.Gist.list}\pysiglinewithargsret{\bfcode{list}}{\emph{user=None}}{}
Get user's gists
\begin{quote}\begin{description}
\item[{Parameters}] \leavevmode
\textbf{user} (\emph{str}) -- Username

\item[{Returns}] \leavevmode
A {\hyperref[result::doc]{\emph{Result}}}

\end{description}\end{quote}

If you call it without user and you are authenticated, get the
authenticated user's gists. but if you aren't authenticated get the
public gists

\begin{Verbatim}[commandchars=\\\{\}]
\PYG{n}{gist\PYGZus{}service}\PYG{o}{.}\PYG{n}{list}\PYG{p}{(}\PYG{l+s}{'}\PYG{l+s}{copitux}\PYG{l+s}{'}\PYG{p}{)}
\PYG{n}{gist\PYGZus{}service}\PYG{o}{.}\PYG{n}{list}\PYG{p}{(}\PYG{p}{)}
\end{Verbatim}

\end{fulllineitems}

\index{public() (pygithub3.services.gists.Gist method)}

\begin{fulllineitems}
\phantomsection\label{gists:pygithub3.services.gists.Gist.public}\pysiglinewithargsret{\bfcode{public}}{}{}
Get public gists
\begin{quote}\begin{description}
\item[{Returns}] \leavevmode
A {\hyperref[result::doc]{\emph{Result}}}

\end{description}\end{quote}

\begin{notice}{note}{Note:}
Be careful iterating the result
\end{notice}

\end{fulllineitems}

\index{star() (pygithub3.services.gists.Gist method)}

\begin{fulllineitems}
\phantomsection\label{gists:pygithub3.services.gists.Gist.star}\pysiglinewithargsret{\bfcode{star}}{\emph{id}}{}
Star a gist
\begin{quote}\begin{description}
\item[{Parameters}] \leavevmode
\textbf{id} (\emph{int}) -- Gist id

\end{description}\end{quote}

\begin{notice}{warning}{Warning:}
You must be authenticated
\end{notice}

\end{fulllineitems}

\index{starred() (pygithub3.services.gists.Gist method)}

\begin{fulllineitems}
\phantomsection\label{gists:pygithub3.services.gists.Gist.starred}\pysiglinewithargsret{\bfcode{starred}}{}{}
Get authenticated user's starred gists
\begin{quote}\begin{description}
\item[{Returns}] \leavevmode
A {\hyperref[result::doc]{\emph{Result}}}

\end{description}\end{quote}

\begin{notice}{warning}{Warning:}
You must be authenticated
\end{notice}

\end{fulllineitems}

\index{unstar() (pygithub3.services.gists.Gist method)}

\begin{fulllineitems}
\phantomsection\label{gists:pygithub3.services.gists.Gist.unstar}\pysiglinewithargsret{\bfcode{unstar}}{\emph{id}}{}
Unstar a gist
\begin{quote}\begin{description}
\item[{Parameters}] \leavevmode
\textbf{id} (\emph{int}) -- Gist id

\end{description}\end{quote}

\begin{notice}{warning}{Warning:}
You must be authenticated
\end{notice}

\end{fulllineitems}

\index{update() (pygithub3.services.gists.Gist method)}

\begin{fulllineitems}
\phantomsection\label{gists:pygithub3.services.gists.Gist.update}\pysiglinewithargsret{\bfcode{update}}{\emph{id}, \emph{data}}{}
Update a single gist
\begin{quote}\begin{description}
\item[{Parameters}] \leavevmode\begin{itemize}
\item {} 
\textbf{id} (\emph{int}) -- Gist id

\item {} 
\textbf{data} (\emph{dict}) -- Input. See \href{http://developer.github.com/v3/gists}{github gists doc}

\end{itemize}

\end{description}\end{quote}

\begin{notice}{warning}{Warning:}
You must be authenticated
\end{notice}

\begin{Verbatim}[commandchars=\\\{\}]
\PYG{n}{gist\PYGZus{}service}\PYG{o}{.}\PYG{n}{update}\PYG{p}{(}\PYG{n+nb}{dict}\PYG{p}{(}\PYG{n}{description}\PYG{o}{=}\PYG{l+s}{'}\PYG{l+s}{edited}\PYG{l+s}{'}\PYG{p}{,}
    \PYG{n}{files}\PYG{o}{=}\PYG{p}{\PYGZob{}}\PYG{l+s}{'}\PYG{l+s}{xample.py}\PYG{l+s}{'}\PYG{p}{:} \PYG{p}{\PYGZob{}}
        \PYG{l+s}{'}\PYG{l+s}{filename}\PYG{l+s}{'}\PYG{p}{:} \PYG{l+s}{'}\PYG{l+s}{new\PYGZus{}xample.py}\PYG{l+s}{'}\PYG{p}{,}
        \PYG{l+s}{'}\PYG{l+s}{content}\PYG{l+s}{'}\PYG{p}{:} \PYG{l+s}{'}\PYG{l+s}{import new\PYGZus{}code}\PYG{l+s}{'}\PYG{p}{\PYGZcb{}}\PYG{p}{\PYGZcb{}}\PYG{p}{)}\PYG{p}{)}
\end{Verbatim}

\end{fulllineitems}


\end{fulllineitems}



\paragraph{Comments}
\label{gists:comments-service}\label{gists:comments}\index{Comments (class in pygithub3.services.gists)}

\begin{fulllineitems}
\phantomsection\label{gists:pygithub3.services.gists.Comments}\pysiglinewithargsret{\strong{class }\code{pygithub3.services.gists.}\bfcode{Comments}}{\emph{**config}}{}
Consume \href{http://developer.github.com/v3/gists/comments}{Comments API}

\begin{notice}{note}{Note:}
This service support {\hyperref[services:mimetypes-section]{\emph{MimeTypes}}} configuration
\end{notice}
\index{create() (pygithub3.services.gists.Comments method)}

\begin{fulllineitems}
\phantomsection\label{gists:pygithub3.services.gists.Comments.create}\pysiglinewithargsret{\bfcode{create}}{\emph{gist\_id}, \emph{message}}{}
Create a comment
\begin{quote}\begin{description}
\item[{Parameters}] \leavevmode\begin{itemize}
\item {} 
\textbf{gist\_id} (\emph{int}) -- Gist id

\item {} 
\textbf{message} (\emph{str}) -- Comment's message

\end{itemize}

\end{description}\end{quote}

\begin{notice}{warning}{Warning:}
You must be authenticated
\end{notice}

\begin{Verbatim}[commandchars=\\\{\}]
\PYG{n}{comment\PYGZus{}service}\PYG{o}{.}\PYG{n}{create}\PYG{p}{(}\PYG{l+m+mi}{1}\PYG{p}{,} \PYG{l+s}{'}\PYG{l+s}{comment}\PYG{l+s}{'}\PYG{p}{)}
\end{Verbatim}

\end{fulllineitems}

\index{delete() (pygithub3.services.gists.Comments method)}

\begin{fulllineitems}
\phantomsection\label{gists:pygithub3.services.gists.Comments.delete}\pysiglinewithargsret{\bfcode{delete}}{\emph{id}}{}
Delete a comment
\begin{quote}\begin{description}
\item[{Parameters}] \leavevmode
\textbf{id} (\emph{int}) -- Comment id

\end{description}\end{quote}

\begin{notice}{warning}{Warning:}
You must be authenticated
\end{notice}

\end{fulllineitems}

\index{get() (pygithub3.services.gists.Comments method)}

\begin{fulllineitems}
\phantomsection\label{gists:pygithub3.services.gists.Comments.get}\pysiglinewithargsret{\bfcode{get}}{\emph{id}}{}
Get a single comment
\begin{quote}\begin{description}
\item[{Parameters}] \leavevmode
\textbf{id} (\emph{int}) -- Comment id

\end{description}\end{quote}

\end{fulllineitems}

\index{list() (pygithub3.services.gists.Comments method)}

\begin{fulllineitems}
\phantomsection\label{gists:pygithub3.services.gists.Comments.list}\pysiglinewithargsret{\bfcode{list}}{\emph{gist\_id}}{}
Get gist's comments
\begin{quote}\begin{description}
\item[{Parameters}] \leavevmode
\textbf{gist\_id} (\emph{int}) -- Gist id

\item[{Returns}] \leavevmode
A {\hyperref[result::doc]{\emph{Result}}}

\end{description}\end{quote}

\end{fulllineitems}

\index{update() (pygithub3.services.gists.Comments method)}

\begin{fulllineitems}
\phantomsection\label{gists:pygithub3.services.gists.Comments.update}\pysiglinewithargsret{\bfcode{update}}{\emph{id}, \emph{message}}{}
Update a comment
\begin{quote}\begin{description}
\item[{Parameters}] \leavevmode\begin{itemize}
\item {} 
\textbf{id} (\emph{int}) -- Comment id

\item {} 
\textbf{message} (\emph{str}) -- Comment's message

\end{itemize}

\end{description}\end{quote}

\begin{notice}{warning}{Warning:}
You must be authenticated
\end{notice}

\end{fulllineitems}


\end{fulllineitems}



\subsubsection{Git Data services}
\label{git_data:github-comments-doc}\label{git_data:git-data-services}\label{git_data::doc}\label{git_data:git-data-service}
\textbf{Example}:

\begin{Verbatim}[commandchars=\\\{\}]
\PYG{k+kn}{from} \PYG{n+nn}{pygithub3} \PYG{k+kn}{import} \PYG{n}{Github}

\PYG{n}{gh} \PYG{o}{=} \PYG{n}{Github}\PYG{p}{(}\PYG{n}{user}\PYG{o}{=}\PYG{l+s}{'}\PYG{l+s}{someone}\PYG{l+s}{'}\PYG{p}{,} \PYG{n}{repo}\PYG{o}{=}\PYG{l+s}{'}\PYG{l+s}{some\PYGZus{}repo}\PYG{l+s}{'}\PYG{p}{)}

\PYG{n}{a\PYGZus{}blob} \PYG{o}{=} \PYG{n}{gh}\PYG{o}{.}\PYG{n}{git\PYGZus{}data}\PYG{o}{.}\PYG{n}{blobs}\PYG{o}{.}\PYG{n}{get}\PYG{p}{(}\PYG{l+s}{'}\PYG{l+s}{a long sha}\PYG{l+s}{'}\PYG{p}{)}

\PYG{n}{dev\PYGZus{}branch} \PYG{o}{=} \PYG{n}{gh}\PYG{o}{.}\PYG{n}{git\PYGZus{}data}\PYG{o}{.}\PYG{n}{references}\PYG{o}{.}\PYG{n}{get}\PYG{p}{(}\PYG{l+s}{'}\PYG{l+s}{heads/add\PYGZus{}a\PYGZus{}thing}\PYG{l+s}{'}\PYG{p}{)}
\end{Verbatim}


\paragraph{GitData}
\label{git_data:gitdata}

\paragraph{Blobs}
\label{git_data:blobs-service}\label{git_data:blobs}

\paragraph{Commits}
\label{git_data:gitdata-commits-service}\label{git_data:commits}

\paragraph{References}
\label{git_data:references}\label{git_data:references-service}

\paragraph{Tags}
\label{git_data:tags-service}\label{git_data:tags}

\paragraph{Trees}
\label{git_data:trees-service}\label{git_data:trees}

\subsubsection{Pull Requests service}
\label{pull_requests::doc}\label{pull_requests:pull-requests-service}\label{pull_requests:github-trees-doc}\label{pull_requests:id1}
\textbf{Example}:

\begin{Verbatim}[commandchars=\\\{\}]
\PYG{k+kn}{from} \PYG{n+nn}{pygithub3} \PYG{k+kn}{import} \PYG{n}{Github}

\PYG{n}{gh} \PYG{o}{=} \PYG{n}{Github}\PYG{p}{(}\PYG{n}{user}\PYG{o}{=}\PYG{l+s}{'}\PYG{l+s}{octocat}\PYG{l+s}{'}\PYG{p}{,} \PYG{n}{repo}\PYG{o}{=}\PYG{l+s}{'}\PYG{l+s}{sample}\PYG{l+s}{'}\PYG{p}{)}

\PYG{n}{pull\PYGZus{}requests} \PYG{o}{=} \PYG{n}{gh}\PYG{o}{.}\PYG{n}{pull\PYGZus{}requests}\PYG{o}{.}\PYG{n}{list}\PYG{p}{(}\PYG{p}{)}\PYG{o}{.}\PYG{n}{all}\PYG{p}{(}\PYG{p}{)}
\PYG{n}{pull\PYGZus{}request\PYGZus{}commits} \PYG{o}{=} \PYG{n}{gh}\PYG{o}{.}\PYG{n}{pull\PYGZus{}requests}\PYG{o}{.}\PYG{n}{list\PYGZus{}commits}\PYG{p}{(}\PYG{l+m+mi}{2512}\PYG{p}{)}\PYG{o}{.}\PYG{n}{all}\PYG{p}{(}\PYG{p}{)}
\end{Verbatim}


\paragraph{Pull Requests}
\label{pull_requests:pull-requests}

\paragraph{Pull Request Comments}
\label{pull_requests:pull-request-comments}\label{pull_requests:pull-request-comments-service}

\section{Result}
\label{result::doc}\label{result:id2}\label{result:github-pullrequests-comments-doc}\label{result:result}
Some requests returns multiple {\hyperref[resources::doc]{\emph{Resources}}}, for that reason the
\code{Github API} paginate it and \textbf{pygithub3} too


\subsection{Smart Result}
\label{result:smart-result}\index{Result (class in pygithub3.core.result.smart)}

\begin{fulllineitems}
\phantomsection\label{result:pygithub3.core.result.smart.Result}\pysiglinewithargsret{\strong{class }\code{pygithub3.core.result.smart.}\bfcode{Result}}{\emph{method}}{}
It's a very \textbf{lazy} paginator beacuse only do a real request
when is needed, besides it's \textbf{cached}, so never repeats a request.

You have several ways to consume it
\begin{enumerate}
\item {} 
Iterating over the result:

\begin{Verbatim}[commandchars=\\\{\}]
\PYG{n}{result} \PYG{o}{=} \PYG{n}{some\PYGZus{}request}\PYG{p}{(}\PYG{p}{)}
\PYG{k}{for} \PYG{n}{page} \PYG{o+ow}{in} \PYG{n}{result}\PYG{p}{:}
    \PYG{k}{for} \PYG{n}{resource} \PYG{o+ow}{in} \PYG{n}{page}\PYG{p}{:}
        \PYG{k}{print} \PYG{n}{resource}
\end{Verbatim}

\item {} 
With a generator:

\begin{Verbatim}[commandchars=\\\{\}]
\PYG{n}{result} \PYG{o}{=} \PYG{n}{some\PYGZus{}request}\PYG{p}{(}\PYG{p}{)}
\PYG{k}{for} \PYG{n}{resource} \PYG{o+ow}{in} \PYG{n}{result}\PYG{o}{.}\PYG{n}{iterator}\PYG{p}{(}\PYG{p}{)}\PYG{p}{:}
    \PYG{k}{print} \PYG{n}{resource}
\end{Verbatim}

\item {} 
As a list:

\begin{Verbatim}[commandchars=\\\{\}]
\PYG{n}{result} \PYG{o}{=} \PYG{n}{some\PYGZus{}request}\PYG{p}{(}\PYG{p}{)}
\PYG{k}{print} \PYG{n}{result}\PYG{o}{.}\PYG{n}{all}\PYG{p}{(}\PYG{p}{)}
\end{Verbatim}

\item {} 
Also you can request some page manually
\begin{quote}

Each \code{Page} is an iterator and contains resources:

\begin{Verbatim}[commandchars=\\\{\}]
\PYG{n}{result} \PYG{o}{=} \PYG{n}{some\PYGZus{}request}\PYG{p}{(}\PYG{p}{)}
\PYG{k}{assert} \PYG{n}{result}\PYG{o}{.}\PYG{n}{pages} \PYG{o}{\PYGZgt{}} \PYG{l+m+mi}{3}
\PYG{n}{page3} \PYG{o}{=} \PYG{n}{result}\PYG{o}{.}\PYG{n}{get\PYGZus{}page}\PYG{p}{(}\PYG{l+m+mi}{3}\PYG{p}{)}
\PYG{n}{page3\PYGZus{}resources} \PYG{o}{=} \PYG{n+nb}{list}\PYG{p}{(}\PYG{n}{page3}\PYG{p}{)}
\end{Verbatim}
\end{quote}

\end{enumerate}
\index{get\_page() (pygithub3.core.result.smart.Result method)}

\begin{fulllineitems}
\phantomsection\label{result:pygithub3.core.result.smart.Result.get_page}\pysiglinewithargsret{\bfcode{get\_page}}{\emph{page}}{}
Get \code{Page} of resources
\begin{quote}\begin{description}
\item[{Parameters}] \leavevmode
\textbf{page} (\emph{int}) -- Page number

\end{description}\end{quote}

\end{fulllineitems}

\index{pages (pygithub3.core.result.smart.Result attribute)}

\begin{fulllineitems}
\phantomsection\label{result:pygithub3.core.result.smart.Result.pages}\pysigline{\bfcode{pages}}
Total number of pages in request

\end{fulllineitems}


\end{fulllineitems}



\subsection{Normal Result}
\label{result:normal-result}\index{Result (class in pygithub3.core.result.normal)}

\begin{fulllineitems}
\phantomsection\label{result:pygithub3.core.result.normal.Result}\pysiglinewithargsret{\strong{class }\code{pygithub3.core.result.normal.}\bfcode{Result}}{\emph{method}}{}
It's a middle-lazy iterator, because to get a new page it needs
make a real request, besides it's \textbf{cached}, so never repeats a request.

You have several ways to consume it
\begin{enumerate}
\item {} 
Iterating over the result:

\begin{Verbatim}[commandchars=\\\{\}]
\PYG{n}{result} \PYG{o}{=} \PYG{n}{some\PYGZus{}request}\PYG{p}{(}\PYG{p}{)}
\PYG{k}{for} \PYG{n}{page} \PYG{o+ow}{in} \PYG{n}{result}\PYG{p}{:}
    \PYG{k}{for} \PYG{n}{resource} \PYG{o+ow}{in} \PYG{n}{page}\PYG{p}{:}
        \PYG{k}{print} \PYG{n}{resource}
\end{Verbatim}

\item {} 
With a generator:

\begin{Verbatim}[commandchars=\\\{\}]
\PYG{n}{result} \PYG{o}{=} \PYG{n}{some\PYGZus{}request}\PYG{p}{(}\PYG{p}{)}
\PYG{k}{for} \PYG{n}{resource} \PYG{o+ow}{in} \PYG{n}{result}\PYG{o}{.}\PYG{n}{iterator}\PYG{p}{(}\PYG{p}{)}\PYG{p}{:}
    \PYG{k}{print} \PYG{n}{resource}
\end{Verbatim}

\item {} 
As a list:

\begin{Verbatim}[commandchars=\\\{\}]
\PYG{n}{result} \PYG{o}{=} \PYG{n}{some\PYGZus{}request}\PYG{p}{(}\PYG{p}{)}
\PYG{k}{print} \PYG{n}{result}\PYG{o}{.}\PYG{n}{all}\PYG{p}{(}\PYG{p}{)}
\end{Verbatim}

\end{enumerate}

\end{fulllineitems}



\section{Resources}
\label{resources::doc}\label{resources:resources}


\renewcommand{\indexname}{Index}
\printindex
\end{document}
